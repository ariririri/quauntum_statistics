%==============
%一行目に必ず必要
%文章の形式を定義
%===============
\documentclass[uplatex]{jsbook}
%===============
%パッケージの定義、必要か不明
%===============
%この下4つを加えることで、mathbbが機能した
\usepackage{amsthm}
\usepackage{amsmath}
\usepackage{amssymb}
\usepackage{amsfonts}
%可換図式用パッケージ
\usepackage{amscd}
\usepackage[all]{xy}
\usepackage{tikz-cd}
%リンク用パッケージ
\usepackage[dvipdfmx]{hyperref}
%複数行コメント
%\usepackage{comment}

\usepackage{my-default}
%タイトルデータ
\title{Quauntum Statistics}
\author{ari}
\date{\today}


%===============
%定理環境の設定
%セクション毎
%===============


\begin{document}

% タイトルを出力
\maketitle
% 目次の表示
\tableofcontents
\part{ヒルベルト空間と量子力学}
\chapter{ヒルベルト空間}
ヒルベルト空間と関数解析の基本的な内容を学び実際に量子力学の内容を触れる.
\section{ベクトル空間}
自明なので省略

\section{内積空間}

\begin{screen}
\begin{dfn}
 $\mathcal{H}$を$K$-vector spaceとする.以下の性質を持つものを内積と呼ぶ.
 \begin{itemize}
   \item (正値性) すべての$\phi \in \mathcal{H}$に対し,$(\phi, \phi) \ge 0$
   \item (正定値性) $(\phi, \phi) = 0$の時$\phi = 0$
   \item (線形性) $a,b \in K$で$\psi_a, \psi_b \in \mathcal{H}$に対し,$(\phi, a \psi_a + b\psi_b) = a(\phi, \psi_a) + b(\phi, \psi_b)$
   \item (対称性) $(\phi, \psi) = (\psi, \phi)^*$
 \end{itemize}
\end{dfn}
\end{screen}

\begin{rem}
  $\phi = 0$の時,$(\phi, \phi) = 0$となる. なぜなら線形性から$(0, 0) = 0(0, 0) = 0$よりわかる.
\end{rem}

\begin{rem}
  $(\phi - \psi, \phi - \psi) = (\psi - \phi, \psi - \phi)$
\end{rem}

\begin{rem}
  内積の条件のうち正定値性以外を満たすものを半正定値な内積と呼ばれる.また線形性を左側を用いて定式化する場合も多い.数学の文献では特に多いらしいが,物理ではこの流儀が多い.
\end{rem}


\begin{epl}
$[a, b]$上の複素数値連続関数全体$C[a,b]$はベクトル空間であり,
$f, g \in C[a, b]$に対し,
\begin{equation*}
 (f, g) :=  \int_a^b f^*(x) \cdot g(x) dx
\end{equation*}
\end{epl}
とすると,内積の公理を満たす.

\section{ヒルベルト空間}
内積からノルムを$||x||:= \sqrt{(x, x)}$で定義する.
そこから$x, y$に対し,$d(x, y) = ||x -y||$とすることで,距離が定義されるため,内積空間は距離空間になる.
\begin{lem}
ノルムは距離を定める.ただし、距離とは以下を満たすものである.
\begin{itemize}
    \item $d(x,y) = d(y,x)$
    \item $d(x,x) = 0$
    \item $d(x, y) + d(y, z) \ge d(x,z)$
\end{itemize}
\end{lem}
\begin{proof}
$(x-y, x- y) = (y-x, y-x)$を示す.それは内積の線形性から$(x-y, x-y)= (-1)(y-x, x-y) = (-1)^2(y-x, y-x)$よりわかる.
次に$d(x, x) =0$は自明.最後の三角不等式は
\begin{align*}
(x-y+y-z, x-y+y-z) &= (x-y, x-y) +2Re(x-y, y-z) + (y-z, y-z) \\
                  & \le (x-y, x-y) +2 ||x-y||\cdot ||y-z|| + (y-z)(y-z)  \\
                  & = (||x-y|| + ||y-z||)^2
\end{align*}
\end{proof}

\begin{lem}
 内積は連続である.つまり,$\phi_n \to \phi, \psi_m \to \psi$の時,
 \begin{align*}
  ||\phi_n|| \to ||\phi||  \\
   (\phi_n, \psi_m) \to (\phi, \psi)
 \end{align*}
\end{lem}
\begin{proof}
三角不等式から,$||\phi|| - ||\phi_n|| \le ||\phi_n - \phi|| \to 0$である.
また,
\begin{align*}
|(\phi, \psi) - (\phi_n, \psi_m)| & = |(\phi - \phi_n, \psi) + (\phi_n, \psi - \psi_m)| \\
                                  & \le ||\phi - \phi_n||||\psi|| + ||\phi_n||||\psi - \psi_m|| \to 0
\end{align*}
よりいえる.
\end{proof}

\begin{screen}
\begin{dfn}
 完備な内積空間をヒルベルト空間という.
\end{dfn}
\end{screen}
つまり,内積が定めるノルムについてBanachになる空間である. 一般にBanach空間は内積が定まらないが,実は中線定理を仮定すれば内積が定まる.

\begin{screen}
\begin{dfn}
  $M$を$\mathcal{H}$の部分空間とする.この時,
  \begin{equation*}
  M^{\perp}:= \{x \in \mathcal{H} \mid \forall y \in M, (y, x) = 0\}
  \end{equation*}
\end{dfn}
を直交補空間という.
\end{screen}

\begin{lem}
 $M$を$\mathcal{H}$を部分空間とする.この時$M^{\perp}$は閉部分空間となる.
\end{lem}
\begin{proof}
  $x_n \in M^{\perp}$をコーシー列とする.この時$x_n \to x$となる$x \in H$が存在する.
  また,$\forall y \in M$に対し,$(x_n, y) =0$となり,内積の連続性から$(x_n, y) \to (x, y) =0$となる.
  よって,$x \in M^{\perp}$となるので,閉である.
\end{proof}

正射影を定義する.
$\mathcal{H}$をヒルベルト空間,$M$をその閉部分空間とする.この時
\begin{equation*}
d(x, M) = \mathrm{inf}_{y\in M}d(x,y)
\end{equation*}
とする.ただし,$d(x,y) = ||x -y||$である.

\begin{prop}
 $M$を$\mathcal{H}$の閉部分空間とする.この時,ある$\phi_M \in M$がただ一つ存在し,
 \begin{equation*}
  d(\phi, M) = d(\phi, \phi_M)
 \end{equation*}
となる.
\end{prop}
\begin{proof}
$inf$の定義より,$\phi_n \in M$で$d(\phi, \phi_n) \to d(\phi, M)$となるものが取れる.
この時$\phi_n$は$M$上のコーシー列になる.
なぜなら
\begin{align*}
||\phi_n - \phi_m||^2 & = ||\phi_n - \phi + \phi - \phi_m ||^2 \\
                      & = 2||\phi_n - \phi||^2 + 2 ||\phi - \phi_m||^2 - ||\phi_n - \phi_m - 2\phi||^2 \\
                      & \le 2||\phi_n - \phi||^2 + 2 ||\phi - \phi_m||^2 - 4d^2  \to 0
\end{align*}
よって$M$の完備性から$\phi_n$の収束先を$\phi_M$とすると,内積の連続性から$||\phi - \phi_n|| \to ||\phi - \phi_M ||$となる.
これから確かに$\phi_M$はinfと一致する.
一意性は
\begin{align*}
||\phi_M - \phi_{M'}||^2 & = ||\phi - \phi_M + \phi_{M'} - \phi ||^2 \\
                         & \le 2||\phi - \phi_M||^2 + 2 ||\phi_{M'} - \phi||^2 - 4d^2 = 0
\end{align*}
よりわかる.
\end{proof}

\begin{thm}[正射影定理]
$M$を$\mathcal{H}$の閉部分空間とする.
この時任意の$x \in \mathcal{H}$に対し,$x = \phi + \psi$と表せる($\phi \in M, \psi \in M^{\perp}$)
\end{thm}
\begin{proof}
$x_M$を$x$に対し上の定理を満たす元とする.$x - x_M \in M^{\perp}$を示す.
$(x - x_M, y) = 0$を言えればよい.
$t \in \mathbb{R}$を用い,$||x - x_M - ty||^2 = ||x - x_M||^2 + 2t\mathrm{Re}(x - x_M, y) + t^2 ||y||^2$となり,
距離の最小性から$t\mathrm{Re}(x - x_M, y) + t^2 ||y||^2$は$t$によらず常に正となる.よって,$\mathrm{Re}(x - x_M, y) = 0$となり$it$を取ると同様に$\mathrm{Im}(x - x_M, y) = 0$となる.
よって,0となる.一意性は$y,y' \in M, z, z' \in M^{\perp}$に対し,$y + z = y' + z'$となったすると,$y - y' = z' -z \in M \cap M^{\perp} = \{0\}$となるので,一意性が言える.
\end{proof}
これは$\psi, \phi$ともに射影となる.$||x -  \phi + m'||^2 =  ||\psi||^2 + ||m'||^2$より$\psi$で最小となる.

\chapter{ヒルベルト空間上の線形作用素}
\section{線形作用素}
$\mathcal{H}, \mathcal{K}$をヒルベルト空間とし,$D \subset \mathcal{H}$で$T:D \to \mathcal{K}$が線形写像となる時
$T$は線形であるといい,$T$を$\mathcal{H}$から$\mathcal{K}$の線形作用素という.$D$を定義域といい$D(T)$で表し,像を$R(T)$で表す.

\section{有界線形作用素}
\begin{screen}
\begin{dfn}
 $T: \mathcal{H} \to \mathcal{K}$を線形作用素とする.$C > 0$が存在し,
 \begin{equation*}
  ||T\psi || \le C || \psi ||
 \end{equation*}
となる時,\textbf{有界線形作用素}という.この時,
\begin{equation*}
  ||T|| = \mathrm{sup}_{||\psi|| = 1} \frac{||T\psi||}{||\psi||}
\end{equation*}
を作用素ノルムという.
また,有界線形作用素全体を$B(\mathcal{H}, \mathcal{K})$と表す.
\end{dfn}
\end{screen}

\begin{epl}
複素数値ボレル可測関数$F: \mathbb{R}^d \to \mathbb{C}$に対し,$\lvert F(x) \rvert \le C| \mathrm{a.e.} x \in \mathbb{R}^d$が成り立つ時,\textbf{本質的に有界}という.
このような$C$の下限を$|F(x)|$の\textbf{本質的上限}といい,$||F||_{\infty}$で表す.
この時 $f \in L^2(\mathbb{R}^d)$に対し,
\begin{equation*}
 \int |F(x)f(x)|^2 dx \le ||F||^2_{\infty} \int |f(x)|^2 dx < \infty
\end{equation*}
\end{epl}
なので,
\begin{equation*}
M_Ff(x) = F(x)f(x)
\end{equation*}
によって,$L^2(\mathbb{R}^d) \to L^2(\mathbb{R}^d)$が定義され,これは有界線形作用素となる.
これを\textbf{掛け算作用素}と呼ぶ.

\begin{prop}
$T: \mathcal{H} \to \mathcal{K}$を有界線形作用素とする.この時,
\begin{itemize}
  \item $\phi_n \in D(T)$に対し$\phi_n \to \phi$ならば,$T\phi_n \to T\phi$となる
  \item $D(T) = \mathcal{H}$の時,$\ker T$は閉部分空間である.
\end{itemize}
\end{prop}
\begin{proof}
  \begin{itemize}
    \item  $T, \phi$に分離し,不等式評価すれば良い.
\begin{align*}
    ||T\phi - T\phi_n|| \le ||T|| ||\phi - \phi_n|| \to 0 \\
\end{align*}
    \item
    $(\phi_n)$を$\mathrm{Ker}T$のコーシー列とする.よってその極限$\phi \in \mathcal{H} =D(T)$となる.これが$\phi \in \mathrm{Ker}T$となることを示す.
    $T\phi_n = 0$より,連続性から$T\phi = 0$となるので,言えた.
  \end{itemize}
\end{proof}

\section{有界線形汎関数とリースの表現定理}
$\mathcal{H}$を$K$上のヒルベルト空間とする.$\mathcal{H}$の部分集合から$K$への写像を\textbf{汎関数}という.
$\mathcal{H}$全体を定義域とする.有界線形汎関数の全てからなる集合を$H^*$と表し,\textbf{双対空間}という.


\begin{thm}[リースの表現定理]
$\forall F \in \mathcal{H}^*$に対し, $\phi_F \in \mathcal{H}$で以下を満たすものが唯一存在する.
\begin{itemize}
  \item $F(\phi) = (\phi_F, \phi)$
  \item $||\phi_F|| = ||F||$
\end{itemize}
\end{thm}
$\mathrm{Ker}F = \mathcal{H}$の時は$||F|| = 0$となるので,$\phi_F= 0$を取ればよい.
なので,$\mathrm{Ker}F \neq \mathcal{H}$の時に具体的に条件を満たす$\phi_F$を構成する.
\begin{itemize}
  \item 存在を構成することで示す.
  今$\mathrm{Ker} F^{\perp} \neq {0}$ではない.よって
  $\mathrm{Ker}F^{\perp}$の元$\phi_0$を取る. $\phi \in \mathcal{H}$に対し,$\psi = \phi - F(\phi)F(\phi_0)^{-1}\phi_0 $を取る.
  これは
  \begin{align*}
  F(\psi) & =  F(\phi) - F(F(\phi) F(\phi_0)^{-1}\phi_0)  \\
          & =  F(\phi) - F(\phi)F(\phi_0)^{-1}F(\phi_0)  = 0
  \end{align*}
  より $\psi \in \mathrm{Ker}F$となる.これより$(\phi_0, \psi) =0$となる.よって
  ここで,$\phi_F = F(\phi_0)^* \phi_0 / || \phi_0||^2$とする.$\phi = \psi + F(\phi)F(\phi_0)^{-1} \phi_0$となり,
  \begin{align*}
  (\phi_F, \phi) & = (\phi_F, \psi + F(\phi)F(\phi_0)^{-1} \phi_0)  \\
                 & = F(\phi_0) / || \phi_0||^2 (\phi_0,  F(\phi)F(\phi_0)^{-1} \phi_0) \\
                 & =  F(\phi)
  \end{align*}
 \item 一意性は上の条件を満たす$\phi_F, \phi_F'$に対し$(\phi_F - \phi_F' , \psi) = 0$よりわかる.
 \item  ノルムは不等式両辺評価.
 $F(\psi) = (\phi_F, \psi) \le ||\phi_F|| ||\psi||$より,$\frac{||F(\phi)||}{\||\phi||} \le ||\phi_F||$となる.
 また
 $\ge$は$F(\phi_F) = ||\phi_F||^2$より,$||F|| \ge \frac{||F(\phi_F)||}{||\phi_F||} = ||\phi_F||$
\end{itemize}

\section{ユニタリ作用素とヒルベルト空間の同型}

\begin{screen}
\begin{dfn}
  $U: \mathcal{H} \to \mathcal{K}$が\textbf{ユニタリ作用素}とは
\begin{itemize}
  \item $D(U) = \mathcal{H}$
  \item $R(U) = \mathcal{K}$
  \item $U$は内積を保存する.つまり
  $(U\phi, U\psi)_{\mathcal{K}} = (\phi, \psi)_{\mathcal{H}}$
\end{itemize}
\end{dfn}
\end{screen}
この時$||U\phi|| = ||\phi||$となるので,作用素ノルム$||U|| = 1$となる.
この時,この写像$U$は全単射である.全射は定義からわかり,単射は$U\phi =0$と$||U\phi||=0$が同値なことからわかる.
ヒルベルト空間の内積を含めた構造を保つので,ユニタリ作用素で移り合うものを\textbf{同型}という.
\begin{rem}
 ヒルベルト空間として同型という意味である.
\end{rem}
また,一般に距離空間$X,Y$の間の連続写像$f:X \to Y$が
$d(x_1, x_2) = d(f(x_1), f(x_2))$となる場合,\textbf{等長変換}という.
ユニタリ作用素は$d(a, b) = ||a - b|| = ||Ua -Ub|| = d(Ua, Ub)$となるので,等長変換である.

\begin{lem}
正射影による分解は内積も含め分解する.つまり,ヒルベルト空間として$H \sim M \oplus M^{\perp}$
\end{lem}
\begin{proof}
ベクトル空間として$\mathcal{H} = M \oplus M^{\perp}$であったが,これがヒルベルト空間としての同型となることを示す.
具体的には,$U: \mathcal{H} \to M \oplus M ^{\perp}, \phi \mapsto (\phi_M, \phi_{M ^{\perp}})$が同型となることを示す.
$(\phi_M + \phi_{M ^{\perp}}, \psi_M + \psi_{M  ^{\perp}}) = (\phi_M, \psi_M) + (\phi_M^{\perp}, \psi_M^{\perp})$となる.
これは$(\phi_M, \phi_{M ^{\perp}}) \cdot (\psi_M, \psi_{M^{\perp}})$と一致する.
\end{proof}

このように高々加算個の閉部分ヒルベルト空間の直和で表せる時,それを$\textbf{直交分解}$という.

\begin{screen}
\begin{dfn}
 位相空間$X$が\textbf{可分}とは可算な稠密部分集合が存在することである
\end{dfn}
\end{screen}

\begin{epl}
  $\mathbb{R}$は可分である.
\end{epl}

\subsubsection{ヒルベルト空間の正規直交基底}
この章だけは別の本を参照して示す.
理由は2つある.
一つはヒルベルト空間と量子力学では可分の定義が一般的なものと異なっており、条件が弱いこと
\cite{K} 黒田では、証明に名に選択公理を用いていおり、鮮やかであることある.

\begin{thm}[グラムシュミットの正規直交化法]
 $\{\phi_k\}$を$\mathcal{H}$の要素の集合で,任意の$n$に対して$\phi_1, \ldots \phi_n$が線形独立であるとする.
 この時$\mathcal{H}$の正規直交系$\{\psi_k\}$で,任意の$n$に対し,$\langle \phi_1 \ldots \phi_n \rangle = \langle \psi_1, \ldots, \psi_n\rangle$となるものが存在する.
\end{thm}
\begin{proof}
$M_i := \langle \phi_1, \ldots, \phi_i\rangle$とする.$i =0$の時は$M_i = \{0\}$とする,
$\psi_{i+1}$を$M_{i}^{\perp}$への射影$\phi_{i+1}$への射影のノルムが$1$となるものとする.
つまり$\psi_{i+1}=\frac{P_{M_i^{\perp}}(\phi_{i+1})}{||P_{M_i^{\perp}}(\phi_{i+1})||}$.
この時$\psi$の生成するベクトル空間が$\phi$の生成するベクトル空間と一致することを示したい.
それは$\psi_{i+1} \neq 0$かつ$M_i$の直交補空間の元なので$\psi_{i+1} \notin M_{i}$と
$\phi_{i+1} - ||P_{M_i ^{\perp}}\phi_{i+1}||\psi_{i+1} \in M_{i}$とより,$\psi_{i+1} \in M_{i+1}$となるので,
$M_{i+1}= \langle\psi_1, \ldots, \psi_{i+1}\rangle$となる.$\phi_{i+1}$は$M_i$の直交補空間の元なので,$M_i$の全ての元と直交し,
$||\phi_i|| =1$となるので,証明された.
\end{proof}

\begin{thm}
 可分なヒルベルト空間$\mathcal{H}$は高々加算個の正規直交基底を持つ.
\end{thm}
\begin{proof}
可分であることから$\mathcal{H}$の稠密な可算集合$\{\phi_k\}$が存在する.
これから,以下の操作をすることで線形独立な集合を作る.$M_{i}:= \langle\phi_1, \ldots, \phi_i\rangle$とする.
$\phi_i \in M_{i-1}$なら省き、そうでないなら残す.
これを繰り返すことにより,残った$\phi_i$たちは一次独立であることがわかる.
ここからグラムシュミットの正規直交化法により,生成するベクトル空間の次元に一致する正規直交な元の組が取れる.
これら全体の生成する空間の直交補空間は閉であり,元々の元が稠密なので,閉部分空間の次元は0になる.
なぜなら、直交補空間の元$\alpha$に対し,元々の空間上で収束する列$\{\alpha_n\}$が取れ,
$||\alpha_n - \alpha||^2 = ||\alpha||^2 + ||\alpha_n||^2 \to 0$となるので,$||\alpha|| = 0$となる.
直交補空間が0であることから全体に一致することがわかる.
\end{proof}

これはZornの補題を用いても示せる.

\begin{thm}[Zornの補題]
空でない順序集合$\Lambda$に対し,$\Lambda$の任意の全順序集合が上界を持つなら,
$\Lambda$には極大元が存在する.
\end{thm}

\begin{epl}
 例えば$[a,b]$は全順序集合であるが,上界として$b$が取れ,これはを極大元になる.
\end{epl}

\begin{epl}
 例えば$\mathbb{R}$全体は全順序集合であるが,上界を持たず、実際,極大元を存在しない.
\end{epl}

またこの定理は選択公理と同値であることが知られる.
この定理を用いることでも証明が出来,さらにこれを使うことで可分とは限らないヒルベルト空間にも,完全正規直交系が存在することがわかる.

証明の概略は以下の通り.正規直交系全体を包含を用いて順序を定めるとこれはZornの補題の仮定を満たす.
全順序集合に対し,その和集合が上界となる.
上界となるには和集合が正規直交することを示す必要がある.
それは$\phi_i, \phi_j$に対しそれらを含む2つの順序集合が存在し,全順序性からどちらかに含まれていることがわかり,その中で$\phi_i, \phi_j$の正規直交性がわかる.
よって,Zornの補題から,正規直交系全体に極大元が存在することがわかる.
もしこれが全体を生成できないとすると生成する空間の直交補空間の大きさ1の元を取ることにより,より大きな正規直交系が取れるため、極大性に矛盾する.
よって言えた.

\begin{screen}
\begin{dfn}
可分な無限次元ヒルベルト空間$\mathcal{H}$に対し,その完全正規直交基底$\phi_i$を固定する.
この時$\phi \in \mathcal{H}$は$\phi = \sum a_i \phi_i$とかけ,$a_i$を\textbf{フーリエ級数}という.
\end{dfn}
\end{screen}

\section{有界線形作用素}
\subsection{有界線形作用素の定義域の拡大}

\begin{thm}
$T:\mathcal{H} \to \mathcal{K}$が有界線形作用素で$D(T)$が稠密とする.
この時,定義域が全体に一致し,$D(T)$上元の関数と一致するものが存在する.
\end{thm}
\begin{proof}
$\phi_n \in D(T) \to \phi$という列に対し,$T\phi_n$はコーシー列になる.
なぜなら,$T$が有界線形作用素なので,$||T\phi_i - T \phi_j|| \le ||T|| \cdot || \phi_i - \phi_j||$とできるためである.
よって$\phi \mapsto T\phi$とすれば,これは線形になり,元々と一致することもある.(本来はwell-defined性も必要だが,コーシー列2つを交互に並べればコーシー列なのでわかる)
\end{proof}

このことから$D(T)$は閉と思ってよい.そうすると射影作用素を考えることで,直交補空間では全て0とする延長が構成できる.
よって,有界線形作用素の定義域は全域だと思って良い.
また$\mathcal{H}$から$\mathcal{K}$への有界作用素全体を$\mathcal{B}(\mathcal{H}, \mathcal{K})$と表す.
定義域と値域が一致する場合は$\mathcal{B}(\mathcal{H})$とかく.

$\mathcal{B}(\mathcal{H}, \mathcal{K})$には,ノルムにより距離を定義できる.ノルムによる距離に対し.

\begin{lem}
$\mathcal{B} (\mathcal{H}, \mathcal{K})$は完備である.
\end{lem}
\begin{proof}
作用素のコーシー列$T_n$に対し,$T$を$\phi \mapsto \lim T_n\phi$で定める.$T_n - T_m$は有界線形作用素なので
$||T_n \phi - T_m \phi|| \le ||T_n - T_m|| \cdot ||\phi||$となり$0$に収束する.
$\mathcal{K}$の完備性が存在し,上の写像は定義できる.これが線形であることはすぐわかる.
有界であることは$||T\phi|| \le ||(T - T_n)\phi||+||T_n\phi|| \le  \epsilon + ||T_n|| \cdot || \phi||$よりわかる.
また,$||T - T_n|| = \sup_{||\phi||=1} \frac{||(T - T_n)\phi||}{||\phi||} \to 0$となるので$T$は$T_n$の収束先であることがわかる.
\end{proof}

\subsection{有界作用素の無限級数とノイマン級数}
$S_N := \sum_{n=1}^{N} T_n$がノルムの意味である$S \in \mathcal{B}(\mathcal{H}, \mathcal{K})$に収束する時
$\sum_{n=1}^{\infty}T_n$で$S$と表す.

\subsection{ヒルベルト空間の有界作用素の空間の位相}
$(\phi, \psi_n) \to (\phi, \psi)$となる時, $\psi_n$は$\psi$に\textbf{弱収束}するという.
ヒルベルト空間として$\mathcal{B}(\mathcal{H}, \mathcal{K})$とすると,
任意のコーシー列$\{T_n\}$に対し,$T_n\phi$が$\mathcal{K}$上弱収束する場合,$T_n$を弱収束という.
またこれが強収束(通常の意味での収束)する場合,強収束するという.
ノルムで収束する時はノルム収束という.

ノルム収束するならば強収束し,強収束するなら,弱収束する.

\section{非有界作用素}

有界でない作用素を非有界作用素という.
ここでは非有界作用素の例を2つあげる
\begin{epl}
関数$F$が本質的に有界でないとき,$F$による掛け算作用素は非有界となる.
\end{epl}

\begin{epl}
$C_0 ^{\infty}(\mathbb{R}^{d})$上の偏微分作用素は非有界である.
\end{epl}

\section{作用素の拡大と共役作用素}
\subsection{作用素の拡大}
作用素の拡大は定義域の拡大になっているものを指す.
$T, S$を$\mathcal{H}$から$\mathcal{K}$に対する線形作用素とする.
$D(T) \subset D(S)$で$S|_{D(T)}= T$の時,$S$を$T$の\textbf{拡大}という.

\subsection{共役作用素}
一般にBanach空間$\mathcal{X},\mathcal{Y}$に対し,$\phi \in \mathcal{B}(\mathcal{X}, \mathcal{Y})$に対し,
$\mathcal{B}(\mathcal{Y}^{*},\mathcal{X}^{*})$上に$\phi^* \in B(\mathcal{Y}^*, \mathcal{X}^*)$が定義される.
具体的には$\phi ^{*}(f_y) = f_y \circ\phi$という線形作用素が定義される.
これが有界なのは$\phi$の有界性から従う.
\begin{align*}
    \frac{||\phi^*(f_y)||}{||f_y||} &= \frac{ ||f_y \circ \phi ||}{||f_y||}\\
                                &\le \frac{||f_y|| \cdot ||\phi||}{||f_y||}\\
                                &\le  ||\phi||
\end{align*}
また,有界でなくとも定義域が稠密であれば同様に定義できるが,有界性が成り立つとは限らない.

改めて,有界線形作用素$T: \mathcal{H} \to \mathcal{K}$に対し,リースの表現定理から
$\phi \in \mathcal{K}$に対し,$F_{\phi} \in \mathcal{K}^*$が存在し,
$F_{\phi}(T(\psi)) = (\phi,T\psi)_K$となる.
$F_{\phi} \circ T   \in \mathcal{H}^*$を定めるので,
$(F_{\phi} \circ T )(\psi)= (\eta, \psi)$となる元$\eta$がただ一つ存在する.
よって$(\phi, T\psi) = (\eta, \psi)$となるものが唯一存在する.
これから$T^*:\mathcal{K} \to \mathcal{H}$を$\phi \mapsto \eta$で定める.
ここではヒルベルト空間に限定し,非有界な場合も含めてこれで定義する.

つまり,
\begin{align*}
    D(T^*):= \{ \phi \in K \mid \forall \psi \in D(T), \exists \eta\in \mathcal{H} s.t. (\phi, T\psi)_{\mathcal{K}} = (\eta, \psi)_{\mathcal{H}}  \}
\end{align*}
とし,$T^*: \mathcal{K} \to \mathcal{H}, \phi \mapsto \eta$とする.
このような写像は一意に定まる.
なぜなら,$(\eta, \psi) = (\eta', \psi)$とすると,$(\eta - \eta' , \psi) = 0$となり,これが任意の$\psi$に対して成り立つので特に$\psi = \eta- \eta'$を取ると,
$||\eta - \eta'|| = 0$となり,$\eta = \eta'$となる.

この時,以下が成り立つ.
\begin{prop}
 $T \in B(\mathcal{H},\mathcal{K})$に対し,$T^* \in B(\mathcal{K}, \mathcal{H})$であり,$T^{**}=T$で,
 $||T^*|| = ||T||$となる.
\end{prop}
\begin{proof}
  実際に$\phi \in \mathcal{K}$に対し,$F_{\phi}:\mathcal{H} \to \mathbb{K}$を$F_{\phi}(\psi) = (\phi, T \psi)$で定める.
  これは$(\phi, T \psi) \le ||\phi| \cdot ||T \psi|| \le ||T|| \cdot ||\psi||$より,$\psi$のとり方によらず,
  $\frac{||(\phi, T\psi)||}{\psi}$は制限されるので,有界線形作用素になる.
  よってリースの表現定理より$(\phi , T\psi) = F_{\phi}(\psi) = (\eta, \psi)$となる$\eta \in \mathcal{H}$がただ一つ定まる.
  これから
  \begin{align*}
   ||T^* \phi || & = ||\eta|| &  \mbox{ definition of }T* \\
                & = ||F_{\phi}|| & \mbox{リースの表現定理} \\
                & = \mathrm{sup} ||(\phi, T \psi)||  & \mbox{definition of }F_{\psi}\\
                & \le ||\phi|| \cdot ||T \psi|| & \mbox{コーシーシュワルツ} \\
                & \le ||\phi|| \cdot ||T|| \cdot ||\psi||  & \mbox{コーシーシュワルツ} \\
  \end{align*}
  より,有界性が言え,さらに$||T^*|| \le ||T||$が言える.
  $T$を$T^*$と取り替えることにより,
  $(\phi, T\psi) = (T^* \phi, \psi) = (\phi, T^{**}\psi)$となり,一意性から$T^{**} = T$となり,
  また図式を同様にすることで$||T^*||  \le ||T|| = ||T^{**}||$より,一致が言える
\end{proof}

\begin{epl}
$L^2(\mathbb{C})$に対し,掛け算作用素$M_F$は$f(x) \mapsto F(x) f(x)$となるものであった.
掛け算作用素$M_F$の共役作用素は$(f, M_Fg)  = (M_F^*f, g)$となるものであり,
\begin{equation*}
 (f, M_Fg)  = \int f^*(x)F(x)g(x)dx = \int (f(x) F^*(x))^* g(x) dx = (M_{F^*}f, g)
\end{equation*}
となるので,上の一意性から,証明できる.
\end{epl}

\section{閉作用素と可閉作用素}
\begin{screen}
\begin{dfn}
 $\phi: \mathcal{H} \to \mathcal{K}$がclosedとは
 $D(T)$の任意の$\phi_n$が$\phi_n \to \phi \in \mathcal{H}, T\phi_n \to \psi$となる時,$\phi \in D(T)$で$\phi = T\psi$となること.
\end{dfn}
\end{screen}

今ヒルベルト空間でも(実際はバナッハ空間でも)
$||u||_X + ||Tu||_{\mathcal{Y}}$で定めるノルムで$D(T)$は完備になる.
今このノルムでのコーシー列$u_n$を取ると,$Tu_n$も$\mathcal{Y}$のノルムでコーシー列となるので,$\mathcal{Y}$上この極限となる元が存在する.
この時,$u_n$の収束先$u$は閉作用素の定義から$u \in D(T)$となる.よってこれは確かに完備になる.
このノルムをグラフノルムという.ヒルベルト空間やバナッハ空間の場合,逆にこのグラフノルムの完備性からでも閉作用素は定義できる.

\begin{thm}
  \begin{itemize}
    \item $T \in B(\mathcal{X}, \mathcal{Y})$ならば閉作用素
    \item $T$は閉作用素であり,$D(T)$が$\mathcal{X}$上稠密かつ$T$が有界なら$D(T) = \mathcal{X}$となり,特に$T \in B(\mathcal{X}, \mathcal{Y})$となる.
  \end{itemize}
\end{thm}
\begin{proof}
  1つ目の証明$u_n \in X$がコーシー列なら$T(u_n )$はコーシー列になり完備性から収束先$v$が存在するが,収束先の一意性から$Tu = v$となる.
  2つ目は稠密性から$u \in \mathcal{X}$に対し,そこに収束するコーシー列$u_n \in D(T)$が取れる.定義域上での有界性から$Tu_n$は収束するので,閉作用素の定義から$u \in D(T)$となる.
\end{proof}

閉作用素は
\begin{equation*}
 \Gamma(T) = \{(u, Tu) \subset \mathcal{X} \times \mathcal{Y}\}
\end{equation*}
が閉でも定義できる.

\begin{thm}
  閉作用素$T$が1対1ならば$T^{-1}$も閉作用素である.
\end{thm}
$f:\mathcal{X}\times \mathcal{Y} \to \mathcal{Y} \times \mathcal{X}$は同相であり,それで
グラフ$\Gamma(T^{-1})$と$\Gamma(T)$は移りあうので、お互い閉集合となり,閉作用素であることがわかる.

\subsection{three basic principles}

\begin{thm}[一様有界性原理]
$\mathcal{X}$をバナッハ空間,$\mathcal{Y}$をノルム空間とし,$T_{\lambda} \in B(\mathcal{X}, \mathcal{Y})$とする.
$u \in \mathcal{X}$に対し,
\begin{equation*}
 sup ||T_{\lambda}u || < \infty
\end{equation*}
なら$\mathrm{sup}|| T_{\lambda}|| < \infty$となる.
\end{thm}

\begin{thm}
 $T \in B(\mathcal{X}, \mathcal{Y})$に対し,$R(T)= \mathcal{Y}$なら,
$T$は開写像になる.
\end{thm}
閉グラフ定理は述べておく.
\begin{thm}
 $\mathcal{X}, \mathcal{Y}$をBanach spaceとする.$T: \mathcal{X} \to \mathcal{Y}$が閉作用素で$D(T)= \mathcal{X}$なら$T \in B(\mathcal{X}, \mathcal{Y})$となる.
\end{thm}

\begin{cor}
 $T$が$\mathcal{X}$から$\mathcal{Y}$への閉作用素とする.$T$が一対一かつ$R(T) = \mathcal{Y}$ならば$T^{-1} \in B(\mathcal{Y}, \mathcal{X})$となる.
\end{cor}
$T$が一対一対応なので,$T^{-1}$は閉作用素である.そこで$T^{-1}$の定義域は$R(T)=\mathcal{Y}$となるので,閉グラフ原理より有界線形作用素となる.


\section{レゾルヴェントとスペクトル}
\subsection{作用素の固有値、固有ベクトル、固有空間}
関数空間の場合固有値を固有関数という.


\begin{epl}
$C^1_P[0, 2\pi]$上の作用素$p$を
$pf(x) = - i \frac{df(x)}{dx}$で定めるとフーリエ展開は固有関数展開になる
具体的には
$\phi_n(x)= \frac{1}{\sqrt{2\pi}} e^{in x}$とすると
$p \phi_n(x) = n \phi_n(x)$となる.
なので,フーリエ変換は一つの作用素の固有関数達による正規直交基底となる.
\end{epl}

固有値を持たない作用素も存在する.

\begin{epl}
$L^2(\mathbb{R})$に対し,$F(x) = x$をかける掛け算作用素を$M_x$とする.
この時,$M_xf = \lambda f$とすると,$(x - \lambda)(f(x)) = 0$となるので,
$f(x) = 0, \mathrm{a.e.}x.$となる.
\end{epl}

\subsection{レゾルヴェント集合とスペクトル}

有限次元のベクトル空間の場合$\mathrm{Ker}(f - \lambda) \neq 0$と,$f - \lambda$が全単射でないことは同値だったが,無限次元の場合はそうなるとは限らない.
そこで
$T$を$\mathcal{H}$上の閉作用素とする.作用素$T - \lambda$が全単射であり,その逆作用素
\begin{equation*}
  R_{\lambda}(T) = (T - \lambda)^{-1}
\end{equation*}
  が有界であるような$\lambda$全体を$T$の\textbf{レゾルヴェント集合}といい,$\rho(T)$で表す.

\begin{prop}
 $\lambda \in \rho(T)$は$S \in B(\mathcal{H}, \mathcal{H})$で$R(S) \subset D(T)$を満たし
 \begin{equation*}
   (T - \lambda) S = I, S(T - \lambda) \subset  I
 \end{equation*}
が成立することである. この時$S = R_{\lambda}(T)$となる.
\end{prop}
\begin{proof}
$\lambda \in \rho(T)$ならレゾルヴェントの定義により,$S = R_{\lambda}(T)$を取れば条件を満たす.
逆は,$(T - \lambda )S = I$より全射であり,$S(T - \lambda) \subset I$よりのKernelは0になるので言える.
\end{proof}

$\sigma(T): = \mathbb{C} \setminus \rho(T)$を$T$の\textbf{スペクトル}という.
また,$T$の固有値全体を$\rho_p(T)$とかき、\textbf{点スペクトル}という.
有限次元の場合は点スペクトルとスペクトルは一致する.

\begin{prop}
 $\mathcal{B} (\mathcal{H}, \mathcal{K})$の点列$\{T_n\}$に対し,$\sum_{n=1}^{\infty}||T_n|| \le \infty$となる時,
 $\sum T_n$は収束する.
\end{prop}
\begin{proof}
$||S_n - S_m|| =|| \sum T_k|| \le \sum ||T_k|| \to 0$となるので
コーシー列となることがわかる.
\end{proof}

\begin{thm}
$T \in B(\mathcal{X})$で$||T|| < 1$の時,$(1+T)$は全単射であり,
\begin{equation*}
 (1+T)^{-1}  = \sum_{n=0}^{\infty} (-1)^{n}T ^{n}
\end{equation*}
が成立し,$(1+T)^{-1}$は有界線形作用素になる.
\end{thm}
\begin{proof}
 $S:= \sum_{n=0}^{\infty} (-1)^{n}T ^{n}$とすると,絶対収束し,及びノルムが$(1- ||T||)^{-1}$以下になる.
 なぜなら
 \begin{align*}
 ||\sum_{n=0}^{\infty} (-1)^{n}T ^{n}|| \le \sum ||T||^n  \le (1 - ||T||)^{-1}
 \end{align*}
$S_n$は$S$の第n次成分までの和とする.
$(1+T)$の連続性(有界性)から
\begin{equation*}
    \lim (1+T)(S_n) = (1+T)(\lim S_n) = (1+T)S
\end{equation*}
となり$(1+T)S_n = (1+(-1)^{n}T^{n+1})$となるので$(1+T)S = 1$となる
逆も同様なので$S(1+T)=1$となるので,全単射となる.
\end{proof}

\begin{rem}
 作用素が有界線形作用であることと連続であることは同値.
 有界なら収束する時にいくらでも小さくできるので連続.
 逆に連続なら背理法でノルムが有界でなくなるような点列を取ると収束先が複数になり連続性に矛盾.
 例えば十分大きい$n$に対し,$u_n$を$||u_n|| = 1$で$|Tu_n| \ge N$となるようなものが取れる.
 この時に$v_n = u_n / ||Tu_n||$とすると,$|v_n | \to 0$になる.一方で$|Tv_n|| = 1$となる.
 よって$\lim Tv_n \neq 0$となる。一方で$v_n \to 0$なので$Tv_n$は連続性から0になる。よって矛盾する.
\end{rem}

\subsection{レゾルヴェントの基本的な性質}
\begin{thm}
 $T$をヒルベルト空間$\mathcal{H}$上の閉作用素とし,$\rho(T) \neq 0$とする.
 \begin{itemize}
   \item $\lambda_0 \in \rho(T)$に対し,$|\lambda - \lambda_0|< ||R_{\lambda_0}(T) ||^{-1}$となる$\lambda$は$\lambda \in \rho(T)$となる.
   また,
   \begin{equation*}
   R_{\lambda}(T) = \sum_{n=0} ^{\infty} R_{\lambda_0}(T)^{n+1}(\lambda - \lambda_0)^n
   \end{equation*}
   となる.特に$\rho(T)$はopen setとなる.
   \item $D(T)$はdenseとする.この時$\lambda \in \rho(T)$ならば$\lambda^* \in \rho(T^*)$であって,$R_{\lambda}(T)^* = R_{\lambda^*}(T^*)$となる.
 \end{itemize}
\end{thm}
\begin{proof}
  \begin{enumerate}
    \item
$K_{\lambda} = (\lambda - \lambda_0)R_{\lambda_0}$とする
\begin{align*}
(1- K_{\lambda})(T - \lambda_0) & =  (1- \frac{\lambda - \lambda_0}{T - \lambda_0)} ) (T -\lambda_0) \\
    & = T - \lambda_0 - (\lambda - \lambda_0)  \\
    & = T - \lambda
\end{align*}
とできる.
$K_{\lambda}$は有界線形作用素の定数倍なので有界線形作用素であり$||K_{\lambda}|| = |\lambda - \lambda_0|\cdot ||R_{\lambda_0}||$となる.
よって$|\lambda - \lambda_0| < ||R_{\lambda_0}(T)||^{-1}$ならば,$||K_{\lambda}|| < 1$であり,これから$1-K_{\lambda}$は全単射であり,
$(1-K_{\lambda})^{-1} \in B(\mathcal{H})$となる.
これより,$T-\lambda$は全単射であることがわかり,さらに$(T-\lambda)^{-1} = (T - \lambda_0)^{-1}(1-K_{\lambda})^{-1}$となるので,有界線形作用素であることもわかる.
よって$\lambda \in \rho(T)$である.
また,
\begin{equation*}
  (1-K_{\lambda})^{-1}= \sum K^n_{\lambda} = \sum_{n=0}^{\infty} R_{\lambda_0}(T)^n(\lambda - \lambda_0)^n
\end{equation*}
となるので、そのまま代入すれば良い.

\item  共役作用素の存在は(0が必ず入るので)言える.
まず,$T-\lambda$の定義域は稠密になり,$R_{\lambda}(T)$が有界線形作用素で定義域が全域なので,
\begin{equation*}
I = I ^* \supset (R_{\lambda}(T)(T - \lambda))^* =  R_{\lambda}(T)^* (T-\lambda)^* = R_{\lambda}(T)^* (T^* - \lambda^*)
\end{equation*}
となる.同様にすることで,$\lambda^* \in \rho(T^*)$となり$R_{\lambda}(T)^* = R_{\lambda^*}(T^*)$となる.
\end{enumerate}
\end{proof}

ところで

\begin{align*}
R_{\lambda}(T) ((T -\mu) - (T - \lambda))R_{\mu}(T) & = R_{\lambda}(T)(1 - (T-\lambda)R_{\mu}(T)) \\
                                                    & = R_{\lambda}(T) - R_{\mu}(T)
\end{align*}
となる.
この式から
$R_{\lambda}(T)R_{\mu}(T) = R_{\mu}(T)R_{\lambda}(T)$がなりたつ.

\begin{thm}
 $T$をヒルベルト空間 $\mathcal{H}$上の閉作用素とする.
 \begin{enumerate}
   \item $\sigma(T)$は$\mathbb{C}$の閉集合
   \item $T$が有界ならば,$\sigma(T) \subset \{\lambda \in \mathbb{C} \mid |\lambda| \le ||T||\}$
   さらに$\sigma(T) \neq 0$
   \item $U$をユニタリ作用素とすると
   \begin{align*}
    \sigma(UTU^{-1}) = \sigma(T) \\
    \sigma_P(UTU^{-1}) = \sigma_P(T)
  \end{align*}
 \end{enumerate}
\end{thm}

\begin{proof}
\begin{enumerate}
  \item レゾルヴェント集合がopenなので成り立つ
  \item $1 - T/ \lambda$は$|\lambda| > ||T||$の時全単射となるので,$\lambda$はレゾルヴェントとなる.
   $\sigma(T) \neq 0$は一旦認める. 基本方針は全体になったとすると,内積を用いてそこから正則関数が定義できるが、レゾルヴェントの有界性から有界な整関数となり,定数関数となる。特に全てに対して0となるが、それは内積の正定値性に矛盾する.
  \item $U(T - \lambda)U^{-1} = UTU^{-1} - \lambda$より,
  $(T-\lambda)$が全単射と$U(T - \lambda)U$は全単射は同値,
  また,
  $UTU^{-1} - \lambda$にかけると$I$になるので,全単射
  また$\mathrm{Ker}(T-\lambda) \to \mathrm{Ker}(UTU^{-1} - \lambda), x \mapsto Ux$は全単射なので,言えた.
\end{enumerate}
\end{proof}

\begin{epl}
  $L^2([0, 2\pi]$上の線形作用素$p_1$を以下で定める
  $D(p_1) = \{f \mid f \mbox{は絶対連続}, f(0)=0\}$
  $p_1f(x) = - i f'(x) a.e . x \in [0, 2\pi]$
  絶対連続は任意の$\epsilon$に対し,ある$\delta$が存在し,$\sum (x_k,y_k) < \delta$なら$\sum (f(x_k), f(y_k))<\epsilon$となること,
  絶対連続ならほとんど至る所微分可能なので上の写像は定義できる.
  この時,$p_1$は閉作用素であり,$\sigma(p_1)=0$となる.
\end{epl}
$(X, B, \mu)$を測度空間とし,$f:X \to \mathbb{C}$に対し$\int |f(x)|^2 d\mu$となる$f$全体を$L^2(X)$と表す.
この空間に$(f, g):= \int f(x)^*g(x) d\mu$で内積を定める.(実際にこれが内積になるにはほとんど至る所0による同一視が必要)

今$ \int (f, p_1g) = \int -if(x)^* g'(x) = [-i f^*g] - \int -i f'^* g$であり,
$f \in C_0^{\infty}(X)$に対し,その微分は$C_0^{\infty}(X)$であり,積分をうまく取ることにより,
$C_{0}^{\infty} \subset D(p_1^*)$が言える.
この時,$C_0^{\infty}$は$L^2$空間上Denseなので...なんだっけか?


\begin{epl}
  $L^2([0, 2\pi]$上の線形作用素$p_2$を以下で定める.
  \begin{align*}
  D(p_2) = C^1[0, 2\pi] \\
  p_2f(x) = - i f'(x) a.e . x \in [0, 2\pi]
  \end{align*}
\end{epl}

\chapter{ルベーグ積分の基本}
ルベーグ積分の一つの目標は極限との相性の良さである.
特に典型的には以下の3定理で表される.

\begin{thm}[単調収束定理]
  $f_1, \ldots$が非負可測関数列でほとんど至るところの$x$で
  \begin{equation*}
   f_1(x) \le f_2(x) \le   \to f(x)
  \end{equation*}
とする.この時,$lim \int_E f_n = \int_E f$となる.
つまり積分してからlimitを取ることとlimitをとってから積分することが一致している.
\end{thm}

\begin{thm}[ファトゥの補題]
  $f_1, \ldots$が非負可測関数列とする.この時
  \begin{equation*}
   \int \underline{lim} f_n \le  \underline{lim} \int f_n
  \end{equation*}
\end{thm}

\begin{thm}[ルベーグの収束定理]
  $f_1, \ldots$が可測関数列でほとんど至るところの$x$で$\lim f_n(x)=f(x)$とする.ある可積分関数$g$が存在し,
  \begin{equation*}
   |f_n(x)| \le g(x), \mathrm{a.e} \mbox{ } x\in E, n=1,2
  \end{equation*}
  とする.この時,$f$は可積分で,$\lim \int_E f_n = \int_E f$となる.不等式評価と元の関数から示せる.
\end{thm}

これらは各点収束で議論できることに注意されたい.
例えば連続関数列の極限は一般に連続とは限らない.
\begin{epl}
  $f_n(x) = x^n (x \in [0, 1])$とする.この時$f_n(x)$は連続関数で,$x \neq 1$の時$\lim f_n(x) = 0$,$x =1$の時$\lim f_n(x) = 1$となり,極限は不連続な関数になる.
  このように極限では連続性は保たれないが,ルベーグ積分の意味でも可積分性は保たれる事が多い(多いというのは一定の上限が必要という意味).
\end{epl}

ルベーグ積分の一つのポイントは測度0集合の概念である.関数の中で挙動のおかしい少数部分を測度0集合におしこめる理論でもある(思想的な発言).

\section{測度}

\subsection{測度の定義}

\begin{screen}
\begin{dfn}
$X$を集合とし,$M$を$X$上の集合族とする.
\begin{itemize}
   \item $\emptyset \in M$
   \item $A \in M \Rightarrow X \setminus A \in M$,
   \item $A,B \in M$に対し,$A \cap B \in M$
\end{itemize}
となる時\textbf{有限加法的集合族}、あるいは\textbf{集合代数}という.
さらに集合代数であって,$A_i \in M$に対し,$\cup_{i=1}^{\infty} A_i \in M$となる時,$\sigma$-\textbf{代数}という.
$\sigma$-代数の組$(X, M)$を可測空間といい、この時$M$の元を可測集合という.
\end{dfn}
\end{screen}

\begin{screen}
\begin{dfn}
 $(X, M)$を可測空間とする.$\mu(\emptyset)=0$となる$\mu: M \to [0, \infty]$で
 \begin{equation*}
  \mu(\sum_{n=1}^{\infty} A_n )=\sum \mu(A_n)
\end{equation*}
を$X$上の測度という.$(X, M ,\mu)$を測度空間という.
\end{dfn}
\end{screen}

\begin{screen}
\begin{dfn}
  $X$を位相空間とする.$X$の開集合を全て含む最小の$\sigma$-代数をボレル集合という.
  また、ボレル集合上定義された測度をボレル測度という
\end{dfn}
\end{screen}

\subsection{ルベーグ外測度}
ルベーグ可測集合はボレル集合よりも真に大きくなる.そこでここではルベーグ外側度を定義する.

\begin{screen}
\begin{dfn}
  $I=(a,b),[a, b),(a,b],[a,b]$に対し,$|I|=b-a$で定める.
  $A \subset \mathbb{R}$に対し
  \begin{equation*}
   m^*(A)= \mathrm{inf} \left\{  \sum_{j=1}^{\infty} |I_j| \mid I_j \mbox{は開区間で} A \subset \cup I_j \right\}
  \end{equation*}
  を\textbf{ルベーグ外測度}という.
\end{dfn}
\end{screen}

\begin{thm}
  $A, B \subset \mathbb{R}, A_i \subset \mathbb{R}$とする.
  \begin{enumerate}
    \item $m^*(\emptyset) = 0$
    \item $A \subset B$なら,$m^*(A) \le m^*(B)$
    \item $m^*(\cup A_j) \le \sum m^*(A_j)$
    \item $x \in \mathbb{R}$,$m^*(A+x) = m^*(A)$
    \item 区間$I$に対し,$m^*(I) = |I|$
  \end{enumerate}
\end{thm}
\begin{proof}
  \begin{enumerate}
    \item 空集合は全ての開集合の含まれるので,infは0になる.
    \item $B$を被覆する$I_j$に対し,$A \subset \cup I_j$となるので,
    $m^*(A) \le m^*(B)$
    \item 真面目に議論すれば示せる.
  \end{enumerate}
\end{proof}


\subsection{ルベーグ可測集合}
\begin{screen}
\begin{dfn}
 $E \subset \mathbb{R}$とする.任意の$A \subset \mathbb{R}$に対し,
 \begin{equation*}
 m^*(A) = m^*(A \cap E) + m^*(A \cap E^c)
 \end{equation*}
 となる時,$E$を\textbf{ルベーグ可測集合}という.$\mathcal{L}$でルベーグ可測集合全体を表す.
\end{dfn}
\end{screen}


\begin{screen}
\begin{dfn}
 $\mathcal{L}$上のルベーグ外測度を\textbf{ルベーグ測度}という.
\end{dfn}
\end{screen}

\begin{rem}
  ルベーグ測度の構成方法としては内測度を定義し,外測度と内測度が一致する時で定める方法もある.
  $m_*(E) = |I| - m^*(I \setminus E)$で定め,これをルベーグ内測度という.
\end{rem}

\begin{thm}
  \begin{itemize}
    \item 任意の区間$I$はルベーグ可測
    \item $\mathcal{L}$は$\sigma$-代数
    \item $E \in \mathcal{L}$なら$E +x \in \mathcal{L}$となる.
    \item $m^*(E)=0$なら$E \in \mathcal{L}$
  \end{itemize}
\end{thm}


\begin{thm}
  \begin{enumerate}
    \item ルベーグ測度$m: \mathcal{L} \to [0, \infty]$は可測空間$(\mathbb{R}, \mathcal{L})$上の測度となる.
    \item $E \in \mathcal{L}$に対し,$m(E+x) = m(E)$となる.
  \end{enumerate}
\end{thm}

これを示すために補題を示す.

\begin{lem}
 $m^*(E)=0$なら$E \in \mathcal{L}$
\end{lem}
\begin{proof}
これは$m^*(A)=m^*(A \cap E) + m^*(A \cap E^c)$を言えればよい.
外測度の劣加法性から$m^*((A\cap E) \cup  (A \cap E^c) \le m^*(A \cap E) + m^*(A \cap E^c)$となるので,逆を言えればよい.
$m^*((A\cap E) \cup  (A \cap E^c) \ge m^*(A \cap E) + m^*(A \cap E^c)$を示せれば良い.
劣加法性から$m^*(E) = m^*(A \cap E) =0$であり,$m^*(A) \ge m^*(A \cap E^C)$となる.よって,
$m^*(A) \ge m^*(A \cap E^C) + m^*(A \cap E)$よりルベーグ可測集合.
\end{proof}


\begin{lem}
  $E \in \mathcal{L}$に対して,$E+x \in \mathcal{L}$
\end{lem}
\begin{proof}
  \begin{align*}
  m^*(A) &= m^*(A-x \cap E) + m^*(A-x \cap E^c)  & E \in \mathcal{L} \\
         &= m^*(A \cap E +x) + m^*(A \cap (E+x)^c ) & x \mbox{が平行移動不変性より}
  \end{align*}
\end{proof}
よって$E+x \in \mathcal{L}$となる.

\begin{lem}
 $E\in \mathcal{L}$なら$E^c \in \mathcal{L}$,$E_1, E_2 \in \mathcal{L}$なら$E_1 \cup E_2 \in \mathcal{L}$となる.
\end{lem}
\begin{proof}
$m^*(A) = m^*(A \cap E^c) + m^*(A \cap E)$より$E^c \in \mathcal{L}$である.
また
\begin{align*}
   & m^*(A \cap (E_1 \cup E_2)) + m^*(A \cap (E_1 \cup E_2)^c) \\
 \le & m^*(A \cap E_1 ) + m^*(A \cap E_1^c \cap E_2) + m^*(A \cap (E_1 \cup E_2)^c)  & \mbox{劣加法性}\\
 = & m^*(A \cap E_1 ) + m^*(A \cap E_1^c \cap E_2) + m^*(A \cap E_1^c \cap E_2^c) \\
 = & m^*(A \cap E_1 ) + m^*(A \cap E_1^c) \\
 = & m^*(A)
\end{align*}
より示せた.
\end{proof}

\begin{lem}
 $E_1 \ldots En \in \mathcal{L}$が互いに素なら,任意の$A \subset \mathbb{R}$に対し,
 \begin{equation*}
  m^*(A \cap \cup_{i=1}^n E_i)  = \sum_{i=1}^n m^*(A \cap E_i)
 \end{equation*}
\end{lem}
\begin{proof}
  $n$に対する帰納法で示す.$n=1$は自明.
  $E_i$は互いに素だから.$m^*(A \cap \cup_{j=1}^n E_j \cap E_n^c) = m^*(A \cap \cup_{j=1}^{n-1}E_j)$
  $m^*(A \cap \cup_{j=1}^n E_j \cap E_n) = m^*(A \cap E_n)$となり,
  $m^*(A \cap \cup E_j) = m^*(A \cap E_n) + m^*(A \cap \cup_{j=1}^{n-1}E_j)$となり,帰納法より
  $m^*(A \cap \cup E_j) = \sum m^*(A \cap E_j)$となる.
\end{proof}

\begin{lem}
  $E_i \in \mathcal{L}$なら$\cup_{n=1}^{\infty}E_n \in \mathcal{L}$となる.さらに$E_1, E_2, \ldots$が互いに素なら,$m(\sum E_n ) = \sum m(E_n)$となる.
\end{lem}
\begin{proof}
$E_j$が互いに素の場合に$ \cup_{n=1}^{\infty} E_n \in \mathcal{L}$を示す.
(集合代数なので、それだけ示せば十分である.なぜなら$E'_n = \cup_{i=0}^{n}E_i \setminus \cup_{i=0}^{n-1} E'_{i}$と取れば良い.)
これは$m^*(A)= m^*(A \cap \cup_{j=1}^n E_j) + m^*(A \cap  (\cup_{j=1}^n E_j)^c)$であり,単調性から
$m^*(A) \ge m^*(A \cap \cup_{j=1}^n E_j) + m^*(A \cap  (\cup_{j=1}^{\infty} E_j)^c)$となる.
$E_i$が互いに素なので $m^*(A) \ge \sum m^*(A \cap  E_j) + m^*(A \cap  (\cup_{j=1}^{\infty} E_j)^c)$となる.
となり,これが任意の$n$に対して成り立つので、
$m^*(A) \ge  \sum_{j=1}^{\infty} m^*(A \cap  E_j) + m^*(A \cap  (\cup_{j=1}^{\infty} E_j)^c)$となる.
右辺の収束は単調増加かつ有界であることから従う.
また劣加法性から
$\sum_{j=1}^{\infty} m^*(A \cap  E_j) + m^*(A \cap  (\cup_{j=1}^{\infty} E_j)^c) \ge  m^*(A \cap  \cup_{j=1}^{\infty} E_j) + m^*(A \cap  (\cup_{j=1}^{\infty} E_j)^c)$となるので,$\cup E_i \in \mathcal{L}$となる.
等式の証明は
この後,$A = \cup E_j$を取ると$m^*(A) = m^*(\cup E_j) = \sum_{j=1}^{\infty} m^*(E_j)$となる.
\end{proof}

\begin{lem}
 任意の区間はルベーグ可測
\end{lem}
\begin{proof}
  $(a, \infty) \in \mathcal{L}$を言えばよく,適当に$A \subset I_i$でinfに十分近い$I_i$を取り,$I_i^{(1)}= I_i \cap (a, \infty)$を取り,$I_i^{(2)}= I_i \cap (a, \infty)^c$を取り、
  $A \cap (a, \infty) ,A \cap (a, \infty)^c$に対し,直接評価し一致を示せる.
\end{proof}

すぐわかることして任意のボレル集合はルベーグ可測
すぐわかることして任意のボレル集合はルベーグ可測


\begin{screen}
\begin{dfn}
 $(X, M, m)$を測度空間とする.
 \begin{enumerate}
   \item  測度0の可測集合の部分集合が全て可測である時 \textbf{完備}という.
   \item $m(X_n) < \infty$となる$X_n \in M$が存在し,$X = \cup X_n$となる時$\sigma$有限という.
 \end{enumerate}
\end{dfn}
\end{screen}

\begin{prop}
ルベーグ測度は次の性質を満たす.
\begin{enumerate}
    \item $A, B \in \mathcal{L}, A \subset B$なら$m(A) \le m(B)$.
    \item $A, B \in \mathcal{L}$の時,$m(A \cup B) + m(A \cap B) = m(A) + m(B)$
    \item $A_1, \ldots, A_n \ldots \in \mathcal{L}$なら$m(\cup A_n) \le \sum m(A_n)$
\end{enumerate}
\end{prop}
\begin{proof}
\begin{enumerate}
    \item $B \setminus A = B \cup (A^c)$よりルベーグ可測で$B = A + B \setminus A$でルベーグ測度の性質から言える.
    \item $C:=A \setminus B, D = B \setminus A$とすると$m(A\cup B) = m(C) + m(D) + m(A\cap B)$であり,$m(A)= m(A\cap B) + m(A\setminus B), m(B)=m(A \cup B) + m(B \setminus A)$より言える.
    \item $\sigma$加法性から明らか.$(B_n = A_n \setminus \cup_{i=1}^{n-1} A_{i}$とすれば,$B_n$は共通部分を持たないので、$m(\cup A_n) = \sum m(B_n) \le \sum m(A_n)$となる.
\end{enumerate}
\end{proof}

\begin{rem}
$A \subset B$の時$m(B\setminus A)$の値は?$A,B$共に無限大の場合はどんな値も取りうる.
例$A= (a, \infty), B = (b, \infty)$とすると$m(B \setminus A) = (a-b)$となるが,$a,b$はどんな値も取りうる...(無限大はとらないが、その場合は$- \infty$から始まる区間を取れば良い.
\end{rem}

\begin{prop}[測度の単調収束定理]
\begin{itemize}
  \item $E_1 \subset E_2 \subset \ldots $なら,$\displaystyle \lim_{j \to \infty} m(E_j) = m(\cup E_j)$
  \item $E_1 \supset E_2 \supset \ldots $なら,$\displaystyle \lim_{j \to \infty} m(E_j) = m(\cap E_j)$
\end{itemize}
\end{prop}

\begin{proof}
$F_j = E_j \setminus E_{j-1}$とすると,$F_j$同士は共通部分を持たないので,以下となる.
\begin{align*}
m(\cup E_j) &= m(\cup F_j)  & \mbox{集合の議論}\\
            & = \sum m(F_j)  & \mbox{σ加法性}\\
            & = \lim_{j \to \infty}\sum_{j=1}^n m(F_j) \\
            & = \lim_{j \to \infty} m(E_j) & \mbox{集合の議論}
\end{align*}
\end{proof}
逆は$F_i = E_i^c$とすると,単調増加なので,$m(X) - \lim m(E_j) = \lim m(X \setminus E_j) =  \lim m(F_j)  = m(\cup F_j) = m(X) - m(\cup E_j)$となるので,言える.


\section{ルベーグ可測関数}
\begin{screen}
\begin{dfn}
  $\mathbb{R}$のルベーグ可測集合$E$と$E$に含まれるルベーグ可測集合からなる可測空間からの写像$f: E \to [-\infty, \infty]$が\textbf{ルベーグ可測}とはルベーグ可測集合の逆像が可測集合となること
\end{dfn}
\end{screen}
ここではルベーグ可測は区間の逆像がルベーグ可測であることさえいえばよいとする(これが上の定義の意味で同値は要確認)

\begin{prop}
 $f,g : E \to \mathbb{R}$が可測なら$cf, f +g, f\cdot g$は可測となる.
\end{prop}
\begin{proof}
  \begin{itemize}
    \item $cf$が可測なのは$c > 0$の時$cf < a$が$f < a/c$と同値になり.$cf^{-1}((-\infty, a)) = f ^{-1}(-\infty, a/c)$となるためである.
    \item $f + g < a$も同様に考えると$(f+g)^{-1}(-\infty , a) = \cup_c (f^{-1}(-\infty, c) \cap g^{-1}(-\infty, a-c))$より可測集合の和集合や共通部分は可測なので言える.積も同様.
  \end{itemize}
\end{proof}

\begin{prop}
 $f: E \to [- \infty, \infty]$が可測で,$f = g \mathrm{a.e}$なら$g$は可測. ただし$a.e$とは$f \neq g$となる$x$全体のなす集合の測度が0を意味する.
\end{prop}
\begin{proof}
  $g(x > a)$で$\{x \mid g(x)> a\}$を表す.
  $\Omega = \{f \neq g\}$とする.$g(x > a) = f(x > a) \cap \Omega^c \cup \Omega \cap g(x > a)$となる.
  左は可測集合同士の共通部分なので可測.右は測度0の集合の部分集合なので可測.(ルベーグ測度の完備性を利用)
\end{proof}

\begin{prop}
$f:[a,b] \to \mathbb{R}$がリーマン可積分とすると,$f$は可測関数
\end{prop}
\begin{proof}
$f$がリーマン可積分なのは不連続点が測度0集合になることなので,$D$を不連続点の集合,$\Omega = [a,b]\setminus D$とする.
$\Omega$上連続なのと不連続点の集合が測度0であることから
$f(x > a) = f(x > a) \cup \Omega^c \cup \Omega \cap f(x > a)$と分解すると上と同様の議論で証明できる.
\end{proof}
\subsection{可測関数の収束極限}

\begin{lem}
 $g_n$を$E$上の可測関数列とする.
 \begin{equation*}
  \bar{g}(x) = \mathrm{sup}\{ g_n(x)\} , \underline{g}(x)= \mathrm{inf}\{g_n(x)\}
 \end{equation*}
 で定める関数$\bar{g}(x), \underline{g}(x)$は可測
\end{lem}
\begin{proof}
$\bar{g}(x > a) = \cup_n g_n(x > a)$より可測.infも同様.
\end{proof}

\begin{thm}
 $f_n$が可測関数の列とすると,上極限$\overline{\lim} f_n$,下極限$\underline{\lim} f_n$は可測関数になる.
\end{thm}
\begin{proof}
上極限は$\mathrm{inf}_n \mathrm{sup}_{k \ge n} f_k$なので上の補題から可測性が言える.下極限も同様.
\end{proof}
\begin{rem}
 特に上極限と下極限が一致する、つまり極限が存在する時,極限は可測関数になる.(収束の意味での完備性)
 これはリーマン積分可能な関数やボレル可測な関数の列の場合は成り立たなかった.(測度0集合の部分集合が全てルベーグ可測集合であることが本質.)
\end{rem}

\begin{screen}
\begin{dfn}
 値域が有限な可測集合を\textbf{単関数}という.
 単関数$\phi(x) = \sum a_j \xi_{E_j}(x)$と表される.($E_j = \phi^{-1}(a_j)$)
 $\phi$の値域を$\{a_1, \ldots, a_n\}$,$E_j = \{x \in E \mid \phi(x)=a_j\}$とした時に
 $\phi(x)= \sum a_j \chi_{E_j}(x)$と表される.これを\textbf{単関数の標準表現}という.
\end{dfn}
\end{screen}

\begin{thm}
 $f$を可測集合$E$上の有界な可測関数とする.この時$|\phi_n(x)| \le |f(x)|$を満たし,$\displaystyle \lim_{n \to \infty} \phi_n(x) =  f(x)$と一様に収束する単関数列$\{\phi_n\}$が取れる.
\end{thm}
$x$のとり方によらず$|(f - \phi_n)(x)| < 2^{-n}$となるような$\phi_n$を構成する
そのために$E_{n,j}= \{x \in E \mid j/2^n < f(x) < (j+1)/ 2^n\}$を考える.
$\phi_n(x) = \sum_{j=1}^{\infty} j/2^n \xi_{E_{n,j}}(x) + \sum_{j=-1}^{-\infty}(j+1)2^{-n}\xi_{E_{n,j}}(x)$とすると,$f(x)$は有界なので、十分大きい$j$や十分小さい$j$に対し$E_{n, j} = \emptyset$となる.
なので、$\phi_n(x)$は単関数の標準表現になっている.
また,常に$|\phi_n(x)| < |f(x)|$より小さく,さらに$|(f - \phi_n)(x)| < 2^{-n}$となるものが構成できる.

\begin{cor}
 $E$上の有界関数$f$が可測であるためには$\lim \phi_n(x) = f(x)$となる単関数列が存在することが必要十分である.
\end{cor}

\begin{screen}
\begin{dfn}
 区間$[a, b]$上の関数$\phi(x)$は$a=x_0 < x_1 < \ldots < x_n = b$が存在して,各小区間$(x_j, x_{j+1})$上ある定数に等しい時,\textbf{階段関数}と呼ばれる.
\end{dfn}
\end{screen}
\begin{rem}
 単関数であって$E_j$が区間の和集合でかけるものを階段関数という.
\end{rem}
\begin{thm}
 $f$を[a,b]上のほとんど至るところ有限な値を取る可測関数とする.どのように小いさな$\epsilon > 0$と$\delta > 0$に対しても
 \begin{equation*}
  m(\{x \in [a, b] \mid |f(x) - h(x)| > \epsilon\}) < \delta
 \end{equation*}
をみたす階段関数$h(x)$が存在する.$M_1 \le f \le M_2$なら,$h$を$M_1 \le h \le M_2$をみたすように取れる.
\end{thm}


\section{ルベーグ積分}
$f:E \to [-\infty, \infty]$を可測関数とした時,
$\lambda_f(t) = m(E(f> t))$で定める.これを$f$の\textbf{分布関数}という.

\begin{lem}
 \begin{itemize}
   \item $\lambda_f(t)$は単調減少関数で$\lambda_f(t) \le m(E)$となる
   \item $f \le M$なら$t \ge M$の時$\lambda_f(t)=0$となる.
   \item $f,g$に対し,$\lambda_{f+g}(t) \le \lambda_f(t/2) + \lambda_g(t/2)$
 \end{itemize}
\end{lem}
\begin{proof}
$(f+g)(x) > t$の時,$f(x)$と$g(x)$のいずれかは$t/2$より大きい.よって, $E(f+g > t) \subset E(f > t/2) \cup E(g > t/2)$となる.
さらに劣加法性から$\lambda_{f+g}(t) \le \lambda_f(t/2) + \lambda_g(t/2)$となる.
\end{proof}

\begin{screen}
\begin{dfn}
 $E$が測度が有限な可測集合,$f$を$E$上の有界な非負可測関数とする.この時$\int_E f dx := \int_0^{\infty} \lambda_f(t)dt$で定義する.
 無限の場合はこれをlimitを取る形で定める.
 これが有限な値を取る時\textbf{ルベーグ積分可能}という.
 $f$が一般の可測関数の場合は$f^+ - f^-$に分割しそれぞれについてルベーグ可積分である時,ルベーグ可積分という.
\end{dfn}
\end{screen}

\begin{prop}
 $0 \le f \le g$なら,$\int f \le \int g$となる.
 $0 \le f, g$がルベーグ可積分なら$f + g$がルベーグ可積分となる.
\end{prop}

\subsection{単関数の積分}

\begin{lem}
$\phi = \sum a_j \chi_{E_j}$を単関数の標準表現とする.
この時

\begin{equation*}
\int \sum a_j \chi_{E_j} = \sum a_j m(E_j)
\end{equation*}
となる
\end{lem}

\begin{equation*}
\lambda_{\chi_{E_j}} = \begin{cases}
 0    & t > 1 \\
 m(E_j)  & 0 < t \le 1
\end{cases}
\end{equation*}
なので,$\int \lambda_{\chi_{E_j}} = \int_0^1 m(E_j) dt$

さらに積分の線形性が(示していないが直接けいさんすることで成り立つことがわかるので)言えた.


単調収束定理を示す.
\subsection{単調収束定理とファトゥの補題}
\begin{thm}[単調収束定理]
$f_1, f_2 \ldots$が非負可測関数列でほとんど至るところ
\begin{equation*}
    f_1(x) \le f_2(x) \ldots \to f(x)
\end{equation*}
とする.この時, $\lim \int_E f_n = \int_E f$となる.
\end{thm}

\begin{proof}
$f_i$も$f$もルベーグ可積分な場合にのみ示す.
$|\int_{0}^{\infty} \lambda_f(t) dt - \int_0^n \lambda_f(t) | < \varepsilon $となる$n$が存在する.
$\lambda_f$がリーマン可積分なので,
$|\int_0^n \lambda_f(t) - \sum _{j=1}^N\lambda_f(t_j)(t_j - t_{j-1})| < \varepsilon$となる$N$が存在する.
$\sum _{j=1}^N\lambda_f(t_j)(t_j - t_{j-1}) = \sum _{j=1}^N m(f > t_j)(t_j - t_{j-1})$であり,測度の一様収束性から
$|\sum _{j=1}^N m(f > t_j)(t_j - t_{j-1}) - \sum _{j=1}^N m(f_n > t_j)(t_j - t_{j-1})| < \epsilon$とできる.
$f_n$についても同様に分割することで
$|\int f_n - \int f |< 5 \varepsilon$とできる.よって示された.
\end{proof}

\begin{thm}[ファトゥの補題]
  $f_1, f_2, \ldots $が非負可測関数とする.この時,
  \begin{equation*}
   \int \underline{\lim} f_n \le \underline{\lim} \int f_n
  \end{equation*}
となる.
\end{thm}
\begin{proof}
$f(x) = \underline{\lim} f_n(x)$とし,$g_n(x) = \inf_{n \ge k} \{ f_k(x) \}$とする.
$f(x) = \sup g_n(x)$より,$g_n(x) \le f(x)$で,$g_1 \le g_2 \ldots$となる.
よって単調収束定理より,$\lim \int g_n = \int f$となる.さらに,$g_n \le f_n$より,
$\int g_n \le \int f_n$となる.よって,

\begin{equation*}
\int f = \lim \int g_n = \underline{\lim} \int g_n  \le \underline{\lim} \int f_n
\end{equation*}
\end{proof}

\subsection{単関数による積分の近似}
\begin{lem}
\begin{enumerate}
    \item $f \ge 0$をほとんど至るところ有限な可測関数とする.この時$m(E(\phi_n \neq 0)) < \infty$を満たす単関数の増大列で,$0 \le \phi_1(x) \le \phi_2(x) \to f(x)$となるものが存在する.
    この時,$\int_E f = \lim \int_E \phi_n$
    \item $f$が可積分とする. $|\phi_n(x)| \le |f(x)|$を満たし,至るところ$f$に収束する単関数の列$\{ \phi_n \}$で$\int_E f = \lim \int \phi_n$を満たすものが存在する.
\end{enumerate}
\end{lem}
\begin{proof}
\begin{equation*}
    \phi_n(x) = \begin{cases}
    0  & |x| > n,  f(x) > n \mbox{のいずれかが成り立つ時} \\
    k/2^n &  |x| \le n, f(x) \in [k/2^n, (k+1)/2^n) \mbox{の時}
    \end{cases}
\end{equation*}
とするとこれは単関数になる.(値域が有限なので)
さらに$\phi_n(x) \le \phi_{n+1}(x)$となる.$\phi_n(x) = k/2^n$の時$f(x) \ge k/2^n = 2k/ 2^{n+1}$となるので
よって$\lim \phi_n(x) = f(x)$が言えればよい.それは,
十分大きい$n$を取れば,$|f(x) - \phi_n(x) | < 1/2^n$となるので言えた.
2.は負の部分と正の部分に分割すればよい.
\end{proof}

\begin{thm}
 $m(E) < \infty$,$f: E \to \mathbb{R}$は有界とする.この時,
 \begin{equation*}
     \inf \left\{ \int_E \phi: \phi \mbox{は単関数} \ge f \right\} = \sup \left\{ \int_E \phi: \phi \mbox{は単関数} \le f\right\}
 \end{equation*}
となることと$f$が可測であることが同値である.
\end{thm}
\begin{proof}
定数を足すことで$f$は非負と思って良い.
$\Rightarrow$を示す. $f$を可測とする.上の式の左辺を$\Sigma_1$,右辺を$\Sigma_2$とする.$\phi_n$を
\begin{equation*}
    \phi_n(x) = 
    k/2^n, f(x) \in [k/2^n, (k+1)/2^n) \mbox{の時}
\end{equation*}
で定め,$\psi_n(x)$を
\begin{equation*}
    \psi_n(x) = k+1/2^n  f(x) \in [k/2^n, (k+1)/2^n) \mbox{の時}
\end{equation*}
とする.
$\phi_n(x) - \psi_n(x) = 1/2^n$となる.
また$E(f(x) < k/2^n)$は可測なので,$E(\phi_n(x) = k/2^n) = E(\phi_n(x) < k+1/2^n) \setminus E(\phi_n(x) < k/2^n$はルベーグ可測集合になる.
$\psi_n$についても同様.よってこれらはルベーグ可積分であり,
よって,$0 \le \Sigma_1 - \Sigma_2 \le \int \psi_n - \int \phi_n = \int 2^{-n} = 2^{-n}m(E)$.これが任意の$n$に対して成り立つので、
$\Sigma_1 = \Sigma_2$となる.

逆を示す.

$\Sigma_2 = \Sigma_1$より,ある$\phi_j \le f \le \psi_j$となる単関数で,$\int \psi_n \to \Sigma_1$,$\int \phi_n \to \Sigma_2$となるものが取れる.
max,minをうまく取ることにより,
\begin{equation*}
    \phi_1(x) \le \phi_2(x) \le \ldots \le f(x) \le \ldots \psi_2(x) \le \psi_1(x)
\end{equation*}

$\phi(x) = \lim \phi_n(x), \psi(x) = \lim \psi_n(x)$とする.これらは可測関数であり$\phi(x) \le f(x) \le \psi(x)$となる.
よって$m(\phi(x) < \psi(x)) = 0$であれば$\phi(x) = f(x) = \psi(x) .\mu a.e$である.この時$f(x)$は可測関数になる.よってこれを示す.
$\Omega_k = E(\psi - \phi > 1/k)$とする.
$\psi_n - \phi_n \ge \phi - \psi \ge 1/k X_{\Omega_k}$であり,
\begin{equation*}
    \int (\psi_n - \phi_n) \ge \int (\psi - \phi) \ge \int 1/k X_{\Omega_k} = 1/k m(\Omega_k)
\end{equation*}
となる.$n \to \infty$の時左辺は0になるので、任意の$k > 0$に対し,$m(\Omega_k) = 0$となるので、$m(\phi(x) < \psi(x)) = 0$となる.
\end{proof}

\subsection{積分の線形性と単調性}

\begin{thm}
$f,g$を$E$上の可積分関数とする.$a,b \in \mathbb{R}$とする.このとき以下が成り立つ.

\begin{itemize}
    \item $\int af + bg = a \int f + b \int g$
    \item $A, B \subset E$が可測で,$A \cap B = \emptyset$ならば,$\int_{A \cup B}f = \int_A f + \int_B f$
    \item $f \le g$とすると, $\int f \le \int g$
\end{itemize}
\end{thm}
\begin{proof}
$f$が非負関数とし,$\lambda_{f}(t) = m(f> t)$とする.このとき$\int f = \int_0^{\infty} \lambda_f(t)$であった.
今$a> 0$の場合だけを示す.他も同様にできる.$af $に対して考えると,$\lambda_{af}(t) = m(af > t) = m(f > t/a) = \lambda_{f}(t/a)$となる.
よって
\begin{equation*}
    \int af = \int \lambda_{af}(t) = \int_{0}^{\infty} \lambda_f(t/a) dt = \int_{0}^{\infty} a\lambda_f (t)dt =  a \int f
\end{equation*}
$f,g$共に非負の単関数の場合を考える.
つまり,$f = \sum a_i \chi_{A_i}, g = \sum b_j \chi_{B_j}$とする.$C_k = A_i \cap B_j$とする.$C_k$同士は互い共通部分を持たず,$\cup C_k = E$となることに注意すると,
このとき$f+g = \sum_{k} (a_i+b_j) C_k$も単関数となる.
このとき,$\int f+g =  \sum m(C_k)(a_i+b_j) = \sum m(A_i) a_i + \sum m(B_j) b_j = \int f + \int g$となる.
一般の場合は
$f = \lim \phi_n, g = \lim \psi_n$となる,単調増加な単関数列$\phi_n, \psi_n$が取れる.すると,$f + g = \lim (\phi_n + \psi_n)$で,これも単調増加な単関数列となる.
すると,単調収束指定理から$\int f + g = \lim \int \phi_n + \psi_n$となる.単関数に対して積分が分解できるので,
$\lim \int \phi_n + \psi_n = \lim \int \phi_n +  \lim \int \psi_n = \int f + \int g$となる.
2つ目は$f =  f_A + f_B$と分解すればよい.$f_A(x)$は$x \in A$のとき,$f(x)$,そうでないとき$0$とする.これはルベーグ可測である.
このとき$\lambda^{A \cup B}_{f_A}(x) = m(f_A > x) = m ( (f > x) \cap A) $
このとき$\lambda^{A}_{f_A}(x) = m(f_A > x \cap A) = m ( (f > x) \cap A) = \lambda^{A}_f(x) = \lambda^{A \cup B}_{f_A}(x)$となる.
よって$\int_{A \cup B} f =  \int_{A \cup B} f_A + \int_{A \cup B} f_B = \int_A f_A + \int_B f_B = \int_A f + \int_B f$となる.
\end{proof}


\begin{prop}[項別積分定理]
 \begin{enumerate}
     \item 可測な関数列$u_1, u_2, ... \ge 0$に対し,$\int_E \sum u_n = \sum \int u_n$
     \item $A_i$を互いに素とする.$f \ge 0$が可測なら$\int_{\sum A_i} f = \sum \int_{A_i} f$
\end{enumerate}
\end{prop}
\begin{proof}
$f_n = \sum_{i=1}^n u_i$とすると,$f_n$は単調増加なので,
\begin{equation*}
\int \sum_{i=1}^{\infty} u_i = \lim_{n \to \infty} \int f_n = \lim_{n \to \infty} \int \sum_{i=1}^n u_i = \lim_{n \to \infty} \sum_{i=1}^n \int u_i 
\end{equation*}
1つ目のイコールは単調収束定理,2つ目は定義,3つ目は積分の加法性である.
\end{proof}

\subsection{変数変換公式}
\begin{thm}
 $\phi: \mathbb{R} \to \mathbb{R}$を$C^1$級で$\phi'(x) > 0$とする.$E \subset \mathbb{R}$に対し,$F = \phi(E)$とする.
 $f : F \to [- \infty, \infty]$とする.このとき,
 \begin{enumerate}
     \item $E$が可測なら,$F = \phi(E)$も可測,$E$がゼロ集合であることと$F$が測度ゼロ集合であることが同値.
     \item $f$が可測なら,$f \circ \phi: E \to [- \infty, \infty]$も可測.
     \item $f$が非負可測なら$\int_F f(x) = \int f(\phi(t)) \phi'(t) dt$となる.
     \item $f$が可積分なら$f\circ \phi$も可積分であり,$\int_F f(x) = \int f(\phi(t)) \phi'(t) dt$となる.
 \end{enumerate}
\end{thm}
\begin{proof}
proof 
\begin{enumerate}
    \item 
$\phi^{-1}$も連続関数であり,特にルベーグ可積分なので, $F$は逆関数$\phi^{-1}$での$E$の逆像なのでルベーグ可測集合である.

$\cup I_j$が有界な集合とする.すると$\cup I_j$を含む有界な閉集合$V$が存在する.
$\phi$は有界閉集合$V$上で最大値$a$,最小値$b$を持つち.$C^1$級なので,$\phi'(x)$も連続関数であり,最大値$c$を持つ.
今$I_j = (a_j,b_j)$とする.
$|\phi(I_j)| = \phi(b_j) - \phi(a_j) = (b_j - a_j) \phi'(c_j)$となる(平均値の定理).
よって$|\phi(I_j)| < c |I_j|$が言えた.

これから$E$が測度ゼロ集合の場合,
$\forall \epsilon > 0$に対し,十分大きい$n$を取れば$\sum |I_j^n | < \epsilon$となる.
よって
$\sum |\phi(I_j^n)| < c \epsilon$となり,$\phi(E)$が測度ゼロとなることがわかる.
逆も同様に言える.
\item 連続関数はルベーグ可測であるため,$\phi$もルベーグ可測になる.ルベーグ可測関数の定義はルベーグ可測集合の逆像がルベーグ可測集合になることであった.
よって$A \subset [- \infty, \infty]$がルベーグ可測とすると,$f^{-1}(A)$はルベーグ可測である.また,$\phi{-1}(f^{-1}(A)$もルベーグ可測である.
よって$(f \circ \phi)^{-1}(A) = \phi^{-1}(f^{-1})(A)$なので,ルベーグ可測集合になるので,言えた.
\item 
$\phi(\mathbb{R}) = (c, d)$とする,$(- \infty \le c \le d \le \infty)$.
$f$を$x \in F^c$に対しては$f(x) = 0$として,$f: \phi(\mathbb{R}) \to [- \infty, \infty]$に延長する.
この時$f$は可測関数になり,$\int_{[c,d] - F} f(x)  = \int_{E^c} f(\phi(t))\phi'(t)dt = 0$となる.
よって,$E = \mathbb{R}$の時に示せば良い.
さらに,$c_n > c, d_n < d$で$c_n \to c, d_n \to d$となる列を取る.
この時,$\phi(a_n) = c_n ,\phi(b_n) = d_n$とかける.
今,$\alpha_n \to - \infty, \beta_n \to \infty$であり,単調収束定理より,
\begin{equation*}
    \int_{c_n}^{d_n} f(x) dx = \int_{\alpha_n}^{\beta_n}  f(\phi(t))\phi'(t)dt
\end{equation*}
が言えれば良い.(実際には
$f_n(x) = \begin{cases} 
 f(x)  &   (x \in [c_n, d_n]\\
 c    & otherwise
\end{cases}$
を取れば良い.


以下の流れで示す
\begin{enumerate}
\item 開区間の定義関数の場合
\item 一般の開集合,閉集合の定義関数の場合
\item 閉集合の無限和でかける集合上の定義関数の場合
\item 一般の可測集合の定義関数の場合
\item 単関数の場合
\item 全体
\end{enumerate}

\begin{itemize}
    \item $f(x)$が開区間$(a,b)$の特性関数のとき$a = \phi(\alpha), b = \phi(\beta)$に対し,
    \begin{equation*}
        \int_{a}^b fdx = \int_{a}^b dx = b-a = \int_{\alpha}^{\beta} \phi'(x) dx
    \end{equation*}
    となる.(有界区間上のルベーグ積分とリーマン積分の一致を使う.)
\end{itemize}
開集合は互いに素な開区間の有限和で表せる.つまり$O = \cup (a_n, b_n), ((a_n, b_n ) \cap  (a_m, b_m) = \emptyset$となるので,
$f$が開集合の定義関数の場合も正しい.
閉集合$K \subset (c_n, d_n)$の定義関数は$(c_n, d_n),K^c$の定義関数の差になる.
つまり,$\chi_K = \chi_{c_n, d_n} - \chi_{K^c}$となるので,ルベーグ積分同士を引くだけなので、一致する.
閉集合の無限和でかける場合,
特に$K_1 \subset K_2 \subset \ldots$が取れ,$L = \cup K_n$とする.この時,$\chi_{K_1} \le \chi_{K_2} \ldots ... \to \chi_L$となるので,単調収束定理より,成り立つ.
またこのような閉集合の無限和でかける集合全体を$\mathcal{F}_{\sigma}$集合という.

一般の可測集合$F$に対し,$m(F \setminus L) = 0$となる$L \in \mathcal{F}_{\sigma}$が存在する.
$N = F \setminus L$とおくと,$\phi^{-1}(N)$も測度0になる.
よって一般の可測集合の場合の定義関数でも成り立つ.

単関数は$f = \sum a_k \chi_{F_K}$とかけるものなので,積分の線形性から成り立つ.
一般の場合は,単関数による増加列を作ることによって,すべて示される.
\end{enumerate}
\end{proof}

\section{ルベーグの収束定理とその応用}

\begin{thm}[ルベーグの収束定理]
$f_1, f_2 \ldots, $を$E$上の可積分関数列でほとんど全ての$x \in E$に対し,$\lim f_n(x) = f(x)$かつある可積分関数$g$が存在し,
\begin{equation*}
    |f_n(x) |  \le g(x)  \mbox{a,e}, x \in E
\end{equation*}
とすると,この時$f$は可積分で,$\lim \int f_n = \int f$となる.
\end{thm}

\begin{proof}
ファトゥの補題を使って示す.
$f$が非負の時のみ示す.(後は$f^+ - f^-$に分割して示せば良い.)

ファトゥの補題から$\int  f \le  \underline{\lim} \int f_n  \le \int g$となる.
また$g - f_n$に対し,ファトゥの補題から
$\int g - f \le \underline{\lim} \int g - f_n$となり,
\begin{align*}
    \int g - f  & \le \underline{\lim} \int g - f_n \\ 
                & =  \int g - \overline{\lim}  \int f_n
\end{align*}
よって$\overline{\lim} \int f_n  \le \int f$となる.
よって$\overline{\lim} \int f_n  \le \underline{\lim} \int f_n$となり,示された.
\end{proof}


\begin{thm}
 $E$を可測集合,$I$を開区間,$Q = E \times I$とする.$(x, t) \in Q$の関数$f(x, t)$は$t$について偏微分可能で,任意に$t$を固定すると,$x$について可積分とする.
 この時ある可積分関数$g$で行かを満たすものが存在するとする.
 \begin{equation*}
     \left\vert \frac{\partial f}{\partial x} (x, t) \right\vert \le g(x)
 \end{equation*}
 この時,$F(t) = \int_E f(x, t)$は微分可能で,$F'(t) = \int_E \frac{\partial f}{\partial t}(x, t) dt$となる
 
\end{thm}
\begin{proof}
\begin{equation*}
    \frac{F(t + h) - F(t)}{h} = \int \frac{f(x , t+h) - f(x, t) }{h}dx
\end{equation*}
$g_h(x) =  (f(x, t+h) - f(x, t)) /h$とする.$h \to 0$の時,$g_h(x) \to \frac{\partial f}{\partial t}(x, t)$となる.
 平均値の定理から
 \begin{align*}
     |g_h(x)| = | \frac{\partial f}{\partial t}(x, t+ \theta h)| \le g(x) 
 \end{align*}
 となる.これにルベーグの収束定理によって,
 \begin{equation*}
 F'(t) = \lim_{h \to 0} \int g_h  = \int \lim_{h \to 0} g_h = \int \frac{\partial f}{\partial t}(x, t)
 \end{equation*}
 よって示された.
\end{proof}


\begin{prop}
$\phi$は可積分関数で$\int \phi = 1$となる.$\phi_{\epsilon} = \epsilon^{-1} \phi(x / \epsilon)$ とすれば
\begin{equation*}
    \lim_{\epsilon \to +0} \int \phi_{\epsilon}(x -a) f(x) dx = f(a)
\end{equation*}
が$x =a$で連続な任意の有界関数$f$に対して成立する.
\end{prop}
\begin{proof}
$|f| < M$とする.$x = a + \epsilon y $とすると,変数変換公式から
\begin{equation*}
    \int \phi_{\epsilon}(x -a)f(x) dx = \int \phi(y) f(a + \epsilon y) dy
\end{equation*}
となる.
$f$は$x =a$で連続だから,$\lim_{\epsilon \to 0} \phi(y) f(a + \epsilon y) = \phi(y) f(a)$となる.
もし$\phi(y)f(a + \epsilon y)$がルベーグの収束定理の仮定を満たすならば,
\begin{equation*}
    \lim_{\epsilon \to 0} \int \phi(y)f(a + \epsilon y ) = \int \lim_{\epsilon \to 0 } \phi(y) f(a + \epsilon y ) = \int \phi(y) f(a) dy = f(a) 
\end{equation*}
となる.
また$|\phi(y) f(a + \epsilon y)|  \le  | M\phi(y)|$となるので、仮定を満たす.
\end{proof}


\begin{thm}[項別積分定理]
 $f_1, f_2, \ldots$を$E$上の可積分関数列,$\sum \int |f_n| < \infty$とする
 この時$\int \sum f_i = \sum \int f_i$となる.
\end{thm}
絶対値については前の項別積分定理で示した.
さらに絶対値はルベーグの収束定理と値が絶対収束することから外せる.

\subsection{積分の強絶対連続性}

\begin{thm}\label{thm:strong conti}
 $f$を$E$上可積分とする.任意の$\epsilon > 0$に対し,ある$\delta >0$が存在し,$m(F) < \delta$を満たす任意の$F \subset E$に対して,$|\int_F f | < \epsilon$となるものが存在する.
\end{thm}
\begin{proof}
可積分関数$f$は単関数近似でき,実際に$|\phi| < |f|$, $| \int_E (f - \phi) < \epsilon / 2 $となる単関数$\phi$が存在する.
$M = max|\phi(x)|$は存在する.よって,$\delta < \frac{2 \epsilon}{M}$とする.$m(F) < \delta$の場合,
\begin{equation*}
    \left|\int_F f \right| \le \left| \int_F \phi \right| + \left| \int (f -\phi) \right| \le \int_F M + \frac{\epsilon}{2} \le m(F)M + \epsilon / 2 < \epsilon
\end{equation*}
となる.
\end{proof}


\begin{lem}
$f$は$[a, b]$上可積分とする.$\forall x \in [a, b]$に対し,$\int_a^x f dt = 0$であれば,ほとんど至る所$f(t) = 0$である.
\end{lem}
\begin{proof}
任意の閉集合$F \subset [a, b]$に対して$\int_F f(t) = 0$を示す.それは,$(x, y) \subset [a, b]$上
$\int_{(x, y)} f = \int_a^y f - \int_a^x f = 0$となる.任意の開集合は互いの素な開区間の和集合で表せるので,開集合$O$上でも$\int_O f dt = 0$となる.
よって$\int_F f dt = \int_{[a, b]} - \int_O f dt = 0$となる.

$E = \{x \in [a, b] \mid  f(x) > 0\}$とする.これが測度0であることを示せば良い.
任意の$\epsilon > 0$に対し上の定理の条件を満たす$\delta$を取る.
可測集合$E$はある閉集合$F \subset E$を用いて$m(E \setminus F) < \delta$とできる.
これから
$\int_E f = \int_{E \setminus F} f + \int_F f < \epsilon$とできる.
任意の$\epsilon$に対して成り立つので,$\int_E f = 0$となる.
$m(E) = 0$が測度ゼロ集合の定義であったが,もし$E$が測度0でなければ,$\int_E f  > 0$(単関数近似すれば明らか)となるので,$E$は測度0となる.
\end{proof}

\section{微分と積分の関係}

\subsection{ビタリの被覆定理}
\begin{screen}
\begin{dfn}
 $E \subset \mathbb{R}$,$\mathcal{I}$は正の長さを持つ区間の集合とする.$x \in E$に対し,$x$を含む長さがいくらでも短い$I \in \mathcal{I}$が取れる時,$\mathcal{I}$は$E$の\textbf{ビタリ被覆}という.
\end{dfn}
\end{screen}

\begin{thm}[ビタリの被覆定理]
$E \subset \mathbb{R}$を外測度が有限な集合,$\mathcal{I}$を$E$のビタリ被覆とする.
この時,$m^{*}(E \setminus \cup_{j=1}^{\infty} I_j) =0$を満たす互いに素な区間の列$I_1, \ldots \in \mathcal{I}$が存在する.
この時,特に$m^*(E \setminus \cup_{j=1}^N I_j) < \epsilon$を満たす互いに素な有限個の区間$I_1, \ldots, I_N \in \mathcal{I}$が存在する.
\end{thm}
\begin{proof}
測度が有限な開集合$E \subset O$を任意にとり,$\tilde{\mathcal{I}} =\{\overline{I} \mid I \in \mathcal{I}, \overline{I} \subset O\}$とする.
任意の$x \in E$に対して$x \in I \subset O$となるいくらでも小さく$I \in \mathcal{I}$が存在するので,$\tilde{\mathcal{I}}$もビタリ被覆になる.よってこれに対して示せばよい.
この時,以下のように$I_1, I_2, \ldots$を選ぶ
\begin{enumerate}
    \item $I_1 \in \tilde{I}$は任意に取る.
    \item $I_1, \ldots, I_n$が互いに素の時,もし$\cup I_i \supset E$ならそこで打ち切る.そうでなければ,ある$x \in E \setminus \cup_{j=1}^{n} I_j$が存在し,
    $\cup I_j$は閉集合なので,$x$を含む$I \in \tilde{I}$で$\cup_{j=1}^n I_j$と交わらないものが取れる.
    具体的には$I_j$が閉集合であることから$d(x, I_j) > 0$であり,$\min d(x, I_j) = a$とすると$d(x, I_{n+1}) < a / 2$となるもをとればよい.(これはビタリ被覆の定義から存在する)
    ここでは,さらに
    $k_n:= \mathrm{sup} \{|I| \mid  I \in \tilde{I}, I \cap  (\cup I_j) = \emptyset\}$とすると,$k_n > 0$となる.
    そこで,$I_{n+1}$を,
    \begin{equation*}
        I_{n+1} \cap (\cup I_j) = \emptyset; k_n/2 \le |I_{n+1}| \le k_n
    \end{equation*}
    となるように取る.
\end{enumerate}
このように取り有限回で終了した場合は自明.
終了しなかった場合を考える.
$\sum_{j=1}^{\infty} I_j \le m(O)$となる.
よって十分に大きい$N$を取れば,$\sum_{j=N+1}^{\infty}|I_j| < \epsilon / 5$となる.
この時
\begin{equation*}
    R := E \setminus \cup_{j=1}^N I_j \subset \cup_{j=N+1}^{\infty} 5I_j
\end{equation*}
を示せばよい.
なぜなら,$m(E \setminus \cup_{j=1}^N I_j) \le \cup_{j=N+1}^{\infty} 5|I_j| < \epsilon$となるからである.

$x \in R$を取る.$x \in I, I \in \tilde{I}$で, $I \cap \cup_{j=1}^N I_j = \emptyset$が取れる.
ただし、$I$は$I_{N+1}, \ldots$とは交わる.そうでなければ,任意の$n$に対して$I \cap (\cup I_j) = \emptyset$となる.したがって,$k_n$の定義から
$|I| \le k_n$となるが,$k_n \to 0$なので,$|I| = 0$となり矛盾する.
$N_0$を$I_j$が初めて$I$と交わる番号とする.
この時,$|I| \le k_{N_0 -1}$であり,$k_{N_0-1}/ 2 \le |I_{N_0}|$となる.
よって$|I|/ 2 \le |I_{N_0}|$となる.$I$と$I_{N_0}$は交わるので,
$I \subset 5I_{N_0}$となる.ゆえに$x \in 5I_{N_0}$となり,言えた.

\end{proof}
\subsection{単調増加関数の微分}

\begin{thm}
 区間$[a,b]$上の単調増加関数$f$はほとんど至るところ微分可能で、 導関数$f'$は可積分である.不等式
 \begin{equation*}
  \int_{\alpha}^{\beta} f'(x)dx \le f(\beta) - f(\alpha)
 \end{equation*}
\end{thm}
が任意の$a \le \alpha \le \beta \le b$で成り立つ.

\begin{proof}
まず,$\int f'$が定義できることを示す.
$f$が全ての微分可能とは限らないため,$f'$は存在するとは限らない.
ただし,$f'$が定義できない点全体の集合が測度0であれば,そこをどんな値で補完しても,$\int f'$は一致する.
そのため$f$がほとんど至る所微分可能であることを示す.
$x \le a$なら$f(x) = f(a)$,$x \ge b$なら$f(x) = f(b)$として$f$を$\mathbb{R}$上で定義する.

\begin{align*}
    D^{\pm}f(x) = \overline{\lim_{h \to 0}}\frac{f(x \pm h) - f(x)}{\pm h}, D_{\pm}(x) = \underline{\lim}_{h \to 0} \frac{f(x \pm h)  - f(x)}{\pm h}
\end{align*}
となる.また定義から$D_+f(x) \le D^+f(x), D_-f(x) \le D^-f(x)$となるので,
\begin{align*}
    f'(x) = \lim_{h \to 0}\frac{f(x+h) - f(x)}{h} 
\end{align*}
が存在することは,$D^+f(x) \le D_-f(x), D^-f(x) \le D_+f(x)$が言えることと同値.
つまり今回は$E^{\pm} := \{ x \in [a, b] \mid D^{\pm}f(x) > D_{\mp} f(x) \}$が測度0であることを言えば良い.
$E_{u,v} := \{x \in [a,b] \mid  D^+f(x) > u> v > D_-f(x)\}$とすると,
$E^+ = \cup_{u, v \in \mathbb{Q} }E_{u, v}$となる.
$\mathbb{Q}$は可算なので,$E_{u, v}$が測度0であることを示せば良い.
$s = m^*(E_{u, v})$とする.
$O \supset E_{u,v}$を$m(O) < s + \epsilon$となるようにとる.
(本当に取れるのかTBD.これが取れるなら,$E_{uv}$はルベーグ可測集合になるような...)
$v > D_-f(x)$より,$f(x) = f(x -h) < hv$となるいくらでも小さい$h$が存在する.
よって
\begin{equation*}
    \mathcal{I} := \{[x-h, x] \mid x \in E_{u,v} , h >0 , [x -h, x] \subset O, f(x) - f(x-h) < hv\}
\end{equation*}
とするとこれは$E_{u, v}$のビタリ被覆である.ビタリの被覆定理から,
$\mathcal{I}$上の互いに素な区間列$\tilde{I}_1 = [x_1-h_1, x_1], ...$を$m(E \setminus \cup \tilde{I}_j) = 0$となるように取れる.
\begin{equation*}
    \sum f(x_j) - f(x_j - h_j) \le \sum v h_j \le v m(O) < v(s + \epsilon)
\end{equation*}
が成り立つ.$I_i:= (x_i-h_1, x_i)$とする.この時, $m(E_{u,v} \setminus \cup I_j) = 0$となる.

$A = E_{u,v} \cap \cup I_j$とする$m^*(A) = m^*(E_{u,v}) =s$である.$y \in A$なら,$D^+f(y) > u$だから,
$f(y+k)  - f(y) > ku$となるいくらでも小さい$k > 0$が存在する.$y \in A$はいずれかの開区間$I_j$に含まれ,$k$を十分に小さくとることにより,
$[y, y+k] \subset I_j$となる.よって
\begin{equation*}
    \mathcal{J} := \{ [y, y+k] \mid y \in A, k> 0, \exists j, [y, y+k] \subset I_j, f(y+k) -f(y) > ku\}
\end{equation*}
は$A$のビタリ被覆になる.前と同じ議論により,
$J_i= [y_1, y_1 +k_1] ...$をうまくとると,$m^*(A \setminus \cup J_i) = 0$となる.よって$m^*(A \cap (\cup J_i) = m^*(A) = s$となる.
また

\begin{equation*}
\sum (f(y_i + k_i)  - f(y_i) \ge  \sum k_i u \ge um^*(A \cap (\cup_{i=1}^{\infty} J_i) = us
\end{equation*}
となる.
これらから$f$の単調増加性に注意すると
($[y_i, y_i+k_i], [y_j, y_j + k_j] \subset [x_k -h_k, x_k]$となることにも注意)
\begin{equation*}
us \le \sum f(y_i + k_i) - f(y_i) ) \le \sum f(x_j) - f(x_j - h_j))  \le v(s + \epsilon)
\end{equation*}
$u > v$なので,$s=0$以外ありえない.
となる.これで積分が定義できることがわかった.
後は頑張って計算するぞ
\end{proof}

\subsection{有界変動関数}
\begin{dfn}
 $[a,b]$の分割$\Delta: a=x_0 < x_1 < \ldots < x_n=b$に対して,
正方向の変動量の和を$P_{\Delta}[a, b] = \sum (f(x_j) - f(x_{j-1}))^+$,負方向の変動量の和の絶対値を$N_{\Delta}[a, b]= \sum (f(x_j) - f(x_{j-1}))^-$,振動量の和$T_{\Delta}[a,b] = P_{\Delta}[a,b] + N_{\Delta}[a, b]$とする.
$P[a,b]: = \sup P_{\Delta}[a, b], N[a,b] := \sup N_{\Delta}[a, b], T[a, b]:= \sup T_{\Delta}[a,b]$とする.$T[a, b] < \infty$の時,$f$は$[a, b]$上\textbf{有界変動}という.
\end{dfn}

\begin{lem}
\begin{enumerate}
    \item $f,g$を$[a,b]$上有界変動とすると$\alpha f+ \beta g$も有界変動である.
    \item 単調関数$f$は有界変動であり,単調増加なら,$f(x) - f(a) = P[a, x],$ 
    \item 単調増加関数の差として表される関数は有界変動である.
\end{enumerate}
\end{lem}
\begin{proof}
\begin{enumerate}
\item $h = \alpha f + \beta g$とする. $T_{\Delta}^{h} = \alpha T_{\Delta}^f + \beta T_{\Delta}^g$となる.
よって$\Delta$のとり方によらず
$T_{\Delta}^{h} \le |\alpha| T^f + |\beta| T^g $となる.よって$T^h\le |\alpha| T^f + |\beta| T^g $となるので、有界変動である.

\item $f$が単調増加の場合だけ示す.$P_{\Delta} = f(b) - f(a)$となる.また,$N_{\Delta} = 0$となる.これは$\Delta$のとり方によらないので,$T= f(b) - f(a)$となり有界変動である.
また,$f(x) - f(a) = P[a,x]$となる.
\item 単調増加関数$f,g$は有界変動であり,$f-g$は有界変動になるので,言えた.
\end{enumerate}
\end{proof}

\begin{lem}
$f$が$[a, b]$上有界変動なら次が成立する.
\begin{enumerate}
    \item  $T[\alpha, \beta] = P[\alpha, \beta] + N[\alpha, \beta]$(これは有界変動でなくても成り立つ)
    \item $f(b) - f(a) = P[a, b] - N[a,b]$
    \item 任意の$a < c < b$に対して,$P[a, b]= P[a, c] + P[c,b]$となる.
\end{enumerate}
\begin{proof}
$P_{\Delta}[a, b] - N_{\Delta}[a, b] = \sum (f(x_j) - f(x_{j-1})) = f(b) - f(a)$となる.
つまり,$P_{\Delta}[a, b] = f(b) - f(a) + N_{\Delta}[a, b]$となるので,supを取ると.
$P[a,b] = f(b) - f(a)  + N[a, b]$となる.よって
$T[a,b]= f(b) - f(a) + 2\sup  N_{\Delta}[a, b] = f(b) - f(a) + 2 N[a, b] =  P[a, b] + N[a, b]$となる.

$\Delta_1, \Delta_2$をそれぞれ,$[a,c], [c,b]$の分割とする.

$P_{\Delta_1} + P_{\Delta_2} = P_{\Delta_1 \cup \Delta_2} \le P[a, b]$となる.よって,$P[a, c] + P[c, b] \le P[a, b]$となる.
逆に$\Delta$を$[a, b]$の分割とする.$\Delta$に$c$を加えたもののうち,$[a, c]$の分割を$\Delta_1$,$[c,b]$の分割を$\Delta_2$とする.
$x_j, x_{j-1} \in \Delta$で$x_{j-1} \ge c <  x_j$とする.
\begin{align*}
\left(f(x_j) - f(x_{j-1})\right)^+ &= (f(x_j) - f(c) + f(c) - f(x_{j-1}))^+ \\
       &\le (f(x_j) - f(c))^+ + (f(c) - f(x_{j-1})^+
\end{align*}
よって,$P_{\Delta}[a, b]  \le P_{\Delta_1}[a, c] + P_{\Delta_2}[c, b]$となる.
これから$P[a, b] \le P[a, c] + P[c, b]$となる.よって
$P[a, b] = P[a, c] + P[c, b]$となる.
\end{proof}
\end{lem}

\begin{thm}
$[a, b]$上の関数$f$が有界変動であることは単調増加関数$f_1,f_2$を用い,$f = f_1 - f_2$と表されることが必要十分である.
\end{thm}
\begin{proof}
有界変動関数$f$に対し,
$f(x) - f(a) = P[a, x] - N[a, x]$
と表され,右辺に出てくる$P[a,x], N[a,x]$は単調増加なので、言えた.
\end{proof}
これから,有界変動関数はほとんどいたる所微分可能であり,$f'(x) = P'[a,x] - N'[a, x]$は可積分である.

\subsection{積分の微分}
\begin{thm}
$f$は$[a, b]$上の可積分関数とする.$F(x) = \int_a^x f(t)dt$は連続かつほとんど至る所微分可能で、$F'(x) = f(x) a.e$が成立する.

\end{thm}


\begin{proof}
$F(x)$が連続であることを示したい.それは定義から任意の$\epsilon$に対しある$\delta > 0$が存在し、$|x - x'| < \delta $に対し$|F(x) - F(x') | < \epsilon$となることを示す.
$|F(x) - F(x')| = |\int_{x'}^x f(t)dt|$なので、\ref{thm:strong conti}より連続であることがわかる.
次に$F$が有界変動関数であることを示す。有界変動であれば、ほとんど至る所微分可能である。

$[a, b]$の任意の分割$\Delta$に対し,

\begin{equation*}
\sum |F(x_j) - F(x_{j-1})| \le \sum \int_{x_{j-1}}^{x_j} |f(t)| dt = \int_a^b |f(t)| dt < \infty
\end{equation*}
となる.よって有界変動であることもわかる.
$F'(x) = f(x) a.e$を示す.

\begin{itemize}
    \item $f$が有界、,$|f(x) | < M$とする.
    $g_n(x) =  n (F(x+ \frac{1}{n}) - F(x)) = n \int_x^{x + 1/n} f(t) dt$とする.
    $\lim_{n \to \infty} g_n(x) = F'(x)$となる、また$|g_n(x)| \le n \int_x^{x + 1/n} M =  M$となる.
    定数関数$M$は可積分なので、ルベーグの収束定理によって,
    
    \begin{equation*}
        \int_a^x F'(t) dt = \lim_{n \to \infty}\int_a^x g_n(t) dt
    \end{equation*}
   となる.
   $\int_a^x g_n(t) dt = n\int_a^x  F(t+ 1/n) - F(t) dt = n \int_{a + 1/n}^{x+1/n}F(t) dt - \int_a^x F(t)dt = n (\int_x^{x+ 1/n} - \int^{a+1/n}_a) F(t) dt$
   となる.よって,
   \begin{equation*}
       \lim_{n \to \infty} n (\int_x^{x+ 1/n} - \int^{a+1/n}_a) F(t) dt = F(x)  - F(a) = \int_a^x f(t) dt
   \end{equation*}
   となる.よって$\int_a^x F'(t)dt - f(t) dt= 0$となる.
   これからほとんど至る所$F'(t) = f(t)$となる.
   \item 一般の場合
   $f$は有界変動なので、$f = f^+ - f^-$と分解して,示す.なので、$f$が単調増加のときに示せれば良い.$f$が単調増加なら$F$は単調増加なので、
   単調増加関数に関するルベーグの定理から$\int_a^x F'(t) dt \le F(x) - F(a)$となる.
   逆向きを示す.
  \begin{equation*}
   f_n(x) := \begin{cases} f(x) ,   & f(x) < n \\ n  & f(x) \ge n \end{cases}
  \end{equation*} 
  とする.$G_n(x) := \int^x_a (f(t) - f_n(t)) dt$
  $F(x) = G_n(x) + \int_a^x f_n(t) dt$.$f_n$は有界だから前の結果より$\int f_n dt ' = f(x)$で$G_n(x)$は単調増加になるので、
  $G'_n(x) \ge 0$となる.
  よって
  \begin{equation*}
      F'(x) = G'_n(x) + f_n(x)  \ge f_n(x) a.e
  \end{equation*}
  となる.
  $\lim_{n\to \infty} f_n(x) = f(x)$より,$F'(x) \ge f(x) $となる.
  よって、この式を積分することによって逆向きが言えた.
\end{itemize}
\end{proof}


\subsection{絶対連続性}

\begin{dfn}
微分がほとんど至る所0に等しい有界変動関数を\textbf{特異関数}という.
連続な特異関数を\textbf{特異連続関数}という.
\end{dfn}

$f$が有界変動な時$g(x) = f(x) - \int_a^xf'(t) dt$は上の定理から$g'(x) = 0$ .a.e .$x$を満たすので,$g(x)$は特異関数である.
この章では可積分関数の積分として表される関数を特徴づける.


\begin{dfn}
区間$[a, b]$上の関数$f$は任意の$\epsilon > 0$に対して,ある$\delta > 0$が存在し,有限個の互いに素な区間$[x_1, y_1] , \ldots ,[x_n, y_n]$が
$\sum_{j=1}^n | x_j - y_j| < \delta$を満たすときは常に$\sum_{j=1}^n | f(x_j) - f(y_j)| < \epsilon$となる時
$[a,b]$上絶対連続という.
\end{dfn}

\begin{epl}
\begin{enumerate}
    \item  $x \sin(1/ \sqrt{x})$は$[0, 1]$上絶対連続であるが,$x \sin(1/ x)$は$[0, 1]$上絶対連続でない
    \item ある$L$が存在して,任意の$x, y$に対して$|f(x) - f(y) | \le L |x-y|$となる時リプシッツ連続という.
    リプシッツ連続なら絶対連続である.
\end{enumerate}
\end{epl}
2のみ示す.
$\delta = \epsilon / L$と取ると$\sum |x_j - y_j| < \delta$の時
$\sum |f(x_j) - f(y_j) | =  \sum L |x_j -y_j < \epsilon$となるので,言えた.

今$f$を$[a, b]$上可積分とする.
\begin{equation*}
    F(x) = F(a) + \int_a^x f(t) dt
\end{equation*}
とするとこれは$[a, b]$上絶対連続である.
それは\ref{thm:strong conti}より言える.
逆に絶対連続仮数は可積分関数の積分として表されることを示す.

\begin{lem}
絶対連続なら有界変動であり,したがってほとんど至る所微分可能
\end{lem}
\begin{proof}
$\epsilon =1$に対し,絶対連続性が成り立つ$\delta$を取る.
幅$|\Delta_0| < \delta$の分割$\Delta_0: a=y_0 < y_1 < \ldots < y_N=b$を取る.
この時$\Delta_0$の任意の細分$\Delta_1$に対して$[y_{j-1}, y_j]$に含まれる$[x_{k-1}, x_k]$ をまとめて
\begin{equation*}
T_{\Delta_1}[a, b] = \sum_{j=1}^N \sum_{[x_{k-1}, x_{k}] \subset [y_{j-1}, y_{j}]}| f(x_k) - f(x_{k-1})|
\end{equation*}
と書くとこれは$N$以下となる.
ゆえにこのように細分との共通部分を取れば常に$N$以下となることがわかるので$f$は有界変動である.
\end{proof}

\begin{lem}
$f$が区間$[a,b]$上絶対連続で$f'(x) =0$a.eなら$f$は定数である.
\end{lem}
\begin{proof}
$E = \{ x \in (a,b) \mid f'(x) = 0 \}$とする.$m([a,b] \setminus E) = 0$となる.
任意に$\epsilon > 0$を取る.$\delta$を絶対連続の式が成り立つように取る.
$x \in E$なら$[x, x+h] \subset (a, b)$で$|f(x + k) - f(x)| < k \epsilon, 0 < \forall k < h$となるいくらでも小さい$h > 0$が存在する.
したがってこのような区間を全て集めた

\begin{equation*}
I = \{ [x, x +h] \subset (a,b) : x \in E, h> 0, |f(x+k) - f(x) | < k\epsilon, 0 < \forall k < h \}    
\end{equation*}
は$E$のビタリ被覆である.
被覆定理から
互いに素な有限個の区間$I_j = [x_j, x_j+ h_j]$を$m([a,b] \setminus \cup I_j) = m(E \setminus \cup I_j) < \delta $と取る.
$[a, b] \setminus \cup I_j$は互いに素な有限個の区間の和集合でその長さの和は$\delta$未満になる.
したがって$f$のその上の振動量の和$\sum |f(x_{j+1}) - f(x_j + h_j)|$は$\epsilon$以下である.
一方で$I_1, \ldots I_N$のとり方から,$f$の$\cup I_j$上の振動量の和は$\sum_{j=1}^N h_j  \epsilon \le (b-a) \epsilon$.となる.
これらを加えることで
$|f(b) - f(a) |  \le \epsilon + \epsilon |b-a|$となる.
$\epsilon$は任意だから$f(b) - f(a) =0$である.
今は$a,b$で示したが、$a,b$をとりかえることで,$f$は定数であることがわかる.
\end{proof}







\chapter{作用素解析とスペクトル定理}

\section{正射影作用素}
正射影定理から$\phi \mapsto \phi_M$が定まる.
この対応を$M$上への正射影作用素という.
これは正射影定理から
$\phi + \psi \mapsto \phi_M + \psi_M$となるので線形であることがわかる.
また$||\psi||^2 \ge ||\phi_M||^2$から有界であることがわかる.

\begin{epl}
  $L^2(\mathbb{R})$に対し,$L^2_+(\mathbb{R})$に対する正射影を求める.単純に分割するのみ.
\end{epl}

$M$に対する正射影作用素をもとに正射影作用素を定義する.
\begin{screen}
\begin{dfn}
 $P^2 = P$,$P^* = P$が成り立つ作用素を正射影作用素という.
\end{dfn}
\end{screen}

\begin{prop}
$P$が正射影作用素の時
 \begin{itemize}
   \item $\phi \in R(P)$に対し,$P\phi = \phi$
   \item $P$は非負
   \item $||P|| \le 1$.特に$||P|| \neq 0$の時,1となる.
   \item $R(P)$は閉部分空間となる.
   \item $P \neq 0, I$の時,$\sigma(P) = \{0, 1\}$
 \end{itemize}
\end{prop}
気頑張って証明するか.

\begin{prop}
$M, N$を$\mathcal{H}$の閉部分空間とする.この時,
$P_M P_N = 0$と$M \perp N$は同値
\end{prop}
$x \in N$の時$P_N(x) = x$となり,$P_M(x) = 0$より$x \in M^{\perp}$となることがわかる.
逆に$M, N$が直交している時,$M \subset N^{\perp}$となる.よって,$P_MP_N = 0$となる.

\section{単位の分解と作用素値汎関数}
\begin{screen}
\begin{dfn}
$\mathcal{B}^d$を$\mathbb{R}^d$のボレル集合全体とする.$P(\mathcal{H})$で正射影作用素全体とし,
$E: \mathcal{B}^{d} \to P(\mathcal{H})$とする.
\begin{itemize}
  \item $E(\emptyset) = 0, E(\mathbb{R}^d) = I$
  \item $B = \cup_{n=1}^{\infty} B_n,B_n \cap B_m = \emptyset (n \neq m) $の時
  \begin{equation*}
  E(B) = s- \lim \sum_{n=1}^N E(B_n)
  \end{equation*}
\end{itemize}
となる時,$E$を\textbf{単位の分割}という.
この収束は強収束である.
\end{dfn}
\end{screen}

ここからすぐわかることとして以下がある.
\begin{lem}
  任意の$B_1,B_2 \in \mathcal{B}^{d}$に対し,$E(B_1 \cap B_2) = E(B_1)E(B_2)$となる.
\end{lem}
\begin{proof}
  $C_1:= B_1 \setminus B_1 \cap B_2, C_2 := B_2 \setminus B_1 \cap B_2$とする.
  $B_{12}:= B_1 \cap B_2$とする
  この時
  \begin{align*}
   E(B_1)E(B_2)& = E(C_1 \cup B_{12})E(C_2 \cup B_{12}) \\
                           & = (E(C_1) +E(B_{12})(E(C_2) + E(B_{12})) \\
                           & = E(C_1)E(C_2) + E(C_1)E(B_{12}) + E(B_{12})E(C_2) + E(B_{12})\\
\end{align*}
となる.よって $B_{12} = \emptyset$の時に$E(B_1)E(B_2) = 0$さえ示せば十分である.
この時$E(B_1) +E(B_2) = E(B_1 \cup B_2)$となる.よって両辺を二乗すると,
$E(B_1) + E(B_1)E(B_2) + E(B_2)E(B_1) + E(B_2) = E(B_1 \cup B_2)$となる.
これから,元の式と合わせると,$E(B_1)E(B_2) + E(B_2)E(B_1) = 0$となる.
これ使い,一つの項が消えるこをを示したい.よって,+ものが一致するように適切に元をかける.
具体的には,左と右から$E(B_1)$を書けると$2 E(B_1)E(B_2)E(B_1) = 0$となる.
よって,$E(B_1)E(B_2)E(B_1) = 0$となるため,$E(B_1)E(B_2) + E(B_2)E(B_1) = 0$に左から$E(B_1)$をかけると,
$E(B_1)E(B_2) +E(B_1)E(B_2)E(B_1) = 0$となるので,$E(B_1)E(B_2) = 0$となる.
\end{proof}


\chapter{量子力学の数学的原理}


\part{作用素環}

\part{量子統計}
あるPDFをベースに数学的にまとめる.

\begin{thm}
$\mathcal{Y} \subset \mathcal{X}$を$C^*$-subalgebraとする.この時,$\sigma_{\mathcal{X}}(B) = \sigma_{\mathcal{Y}}(B)$となる.
以降で,$\sigma(B)$となる.
\end{thm}
\begin{proof}
$B \in \mathcal{Y}$が$\mathcal{X}$上可逆なら,$\mathcal{Y}$上可逆であることを示せばよい.
$B^*=B$の時,$\sigma_X(A) \subset [-||A||, ||A||]$なので,
$\lambda_0 = 2i||B|| \in \rho(B)$となる.
この時,$R(B, \lambda_0) = \lambda_0^{-1}(1 - \frac{B}{\lambda_0})^{-1}$となり,
$||\frac{B}{\lambda_0}|| < 1$よりこれの逆元は$\frac{B}{\lambda_0}$の無限和でかける.
よって完備性から$\mathcal{Y}$の元となる.

なんかここの証明がわからん...
\end{proof}

\begin{thebibliography}{99}
\bibitem{K}  関数解析(黒田)
\end{thebibliography}
\end{document}
